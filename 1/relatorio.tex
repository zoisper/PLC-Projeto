
\documentclass[11pt,a4paper]{report}%especifica o tipo de documento que tenciona escrever: carta, artigo, relatório... neste caso é um relatório
% [11pt,a4paper] Define o tamanho principal das letras do documento. caso não especifique uma delas, é assumido 10pt
% a4paper -- Define o tamanho do papel.

\usepackage[portuges]{babel}%Babel -- irá activar automaticamente as regras apropriadas de hifenização para a língua todo o
                                   %-- o texto gerado é automaticamente traduzido para Português.
                                   %  Por exemplo, “chapter” irá passar a “capítulo”, “table of contents” a “conteúdo”.
                                   % portuges -- específica para o Português.
\usepackage[utf8]{inputenc} % define o encoding usado texto fonte (input)--usual "utf8" ou "latin1

\usepackage{graphicx} %permite incluir graficos, tabelas, figuras
\usepackage{url} % para utilizar o comando \url{}
\usepackage{enumerate} %permite escolher, nas listas enumeradas, se os iems sao marcados com letras ou numeros-romanos em vez de numeracao normal

%\usepackage{apalike} % gerar biliografia no estilo 'named' (apalike)

\usepackage{color} % Para escrever em cores

\usepackage{multirow} %tabelas com multilinhas
\usepackage{array} %formatação especial de tabelas em array

\usepackage[pdftex]{hyperref} % transformar as referências internas do seu documento em hiper-ligações.

%Exemplos de fontes -- nao e vulgar mudar o tipo de fonte
%\usepackage{tgbonum} % Fonte de letra: TEX Gyre Bonum
%\usepackage{lmodern} % Fonte de letra: Latin Modern Sans Serif
%\usepackage{helvet}  % Fonte de letra: Helvetica
%\usepackage{charter} % Fonte de letra:Charter

\definecolor{saddlebrown}{rgb}{0.55, 0.27, 0.07} % para definir uma nova cor, neste caso 'saddlebrown'

\usepackage{listings}  % para utilizar blocos de texto verbatim no estilo 'listings'
%paramerização mais vulgar dos blocos LISTING - GENERAL
\lstset{
	basicstyle=\small, %o tamanho das fontes que são usadas para o código
	numbers=left, % onde colocar a numeração da linha
	numberstyle=\tiny, %o tamanho das fontes que são usadas para a numeração da linha
	numbersep=5pt, %distancia entre a numeração da linha e o codigo
	breaklines=true, %define quebra automática de linha
    frame=tB,  % caixa a volta do codigo
	mathescape=true, %habilita o modo matemático
	escapeinside={(*@}{@*)} % se escrever isto  aceita tudo o que esta dentro das marcas e nao altera
}
%
%\lstset{ %
%	language=Java,							% choose the language of the code
%	basicstyle=\ttfamily\footnotesize,		% the size of the fonts that are used for the code
%	keywordstyle=\bfseries,					% set the keyword style
%	%numbers=left,							% where to put the line-numbers
%	numberstyle=\scriptsize,				% the size of the fonts that are used for the line-numbers
%	stepnumber=2,							% the step between two line-numbers. If it's 1 each line
%											% will be numbered
%	numbersep=5pt,							% how far the line-numbers are from the code
%	backgroundcolor=\color{white},			% choose the background color. You must add \usepackage{color}
%	showspaces=false,						% show spaces adding particular underscores
%	showstringspaces=false,					% underline spaces within strings
%	showtabs=false,							% show tabs within strings adding particular underscores
%	frame=none,								% adds a frame around the code
%	%abovecaptionskip=-.8em,
%	%belowcaptionskip=.7em,
%	tabsize=2,								% sets default tabsize to 2 spaces
%	captionpos=b,							% sets the caption-position to bottom
%	breaklines=true,						% sets automatic line breaking
%	breakatwhitespace=false,				% sets if automatic breaks should only happen at whitespace
%	title=\lstname,							% show the filename of files included with \lstinputlisting;
%											% also try caption instead of title
%	escapeinside={\%*}{*)},					% if you want to add a comment within your code
%	morekeywords={*,...}					% if you want to add more keywords to the set
%}

\usepackage{xspace} % deteta se a seguir a palavra tem uma palavra ou um sinal de pontuaçao se tiver uma palavra da espaço, se for um sinal de pontuaçao nao da espaço

\parindent=0pt %espaço a deixar para fazer a  indentação da primeira linha após um parágrafo
\parskip=2pt % espaço entre o parágrafo e o texto anterior

\setlength{\oddsidemargin}{-1cm} %espaço entre o texto e a margem
\setlength{\textwidth}{18cm} %Comprimento do texto na pagina
\setlength{\headsep}{-1cm} %espaço entre o texto e o cabeçalho
\setlength{\textheight}{23cm} %altura do texto na pagina

% comando '\def' usado para definir abreviatura (macros)
% o primeiro argumento é o nome do novo comando e o segundo entre chavetas é o texto original, ou sequência de controle, para que expande
\def\darius{\textsf{Darius}\xspace}
\def\antlr{\texttt{AnTLR}\xspace}
\def\pe{\emph{Publicação Eletrónica}\xspace}
\def\titulo#1{\section{#1}}    %no corpo do documento usa-se na forma '\titulo{MEU TITULO}'
\def\super#1{{\em Supervisor: #1}\\ }
\def\area#1{{\em \'{A}rea: #1}\\[0.2cm]}
\def\resumo{\underline{Resumo}:\\ }

%\input{LPgeneralDefintions} %permite ler de um ficheiro de texto externo mais definições

\title{Processamento de Linguaguens e Compiladores (3º ano LCC)\\
       \textbf{Trabalho Prático 1}\\ Relatório de Desenvolvimento \\ \textbf{Grupo 17}\\ Problema 3
       } %Titulo do documento
%\title{Um Exemplo de Artigo em \LaTeX}
\author{José Pedro Gomes Ferreira\\ A91636 \and Pedro Paulo Costa Pereira\\ A88062
         \and Tiago André Oliveira Leite\\ A91693
       } %autores do documento
\date{\today} %data

\begin{document} % corpo do documento
\maketitle % apresentar titulo, autor e data


\tableofcontents % Insere a tabela de indice
%\listoffigures % Insere a tabela de indice figuras
%\listoftables % Insere a tabela de indice tabelas


%%%%%%%%%%%%%%%%%%%%%%
%%%%%Capitulo1%%%%%%%%
%%%%%%%%%%%%%%%%%%%%%%


\chapter{Introdução} \label{chap:intro} %referência cruzada

Neste documento vamos explicar a solução que implementamos para a resolução do probelma proposto no ambito da unidade curricular de Processsamento de Linguagens e Compiladores. \\
O problema proposto consiste em analisar o ficheiro de texto $calv$-$users.txt$ e atraves da utilização da linguagem Python e da biblioteca de expressões regulares 're' extrair informação de forma a produzir alguns resultados.\\No desenvolvimento do programa procuramos utilizar o conhecimento adquirido nas aulas, esperando por isso que o resultado final cumpra todos os requisitos. 


%%%%%%%%%%%%%%%%%%%%%%
%%%%%Capitulo2%%%%%%%%
%%%%%%%%%%%%%%%%%%%%%%



\chapter{Problema Proposto} \label{chap:ProbelemaProposto} %capitulo e referencia cruzada

\section{Descrição} \label{sec:Descricao} %seccao e referencia cruzada
Construa um ou vários programas
para processar o texto ' clav-users.txt ' em que campos de informação
têm a seguinte ordem: nome, email, entidade, nível, número de chamadas ao backend, com o intuito de calcular alguns
resultados conforme solicitado a seguir:
\begin{itemize}
\item Produz uma listagem apenas com o nome e a entidade do utilizador, ordenada alfabeticamente por nome;
\item Produz uma lista ordenada alfabeticamente das entidades referenciadas, indicando, para cada uma, quantos utilizadores estão registados;
\item Qual a distribuição de utilizadores por níveis de acesso?
\item Produz uma listagem dos utilizadores, agrupados por entidade, ordenada primeiro pela entidade e dentro desta pelo nome;
\item Por fim, produz os seguintes indicadores:
\begin{enumerate}[1.]
\item Quantos utilizadores?
\item Quantas entidades?
\item Qual a distribuição em número por entidade?
\item Qual a distribuição em número por nível?
\end{enumerate}
\end{itemize}
Para terminar, deve imprimir os 20 primeiros registos num novo ficheiro de output mas em formato Json .

\section{Requisitos} \label{sec:Requesitos}
\begin{itemize}
  \item Utilização da linguagem Python.
  \item Resolver o problema com uso de expressoes regulares.
  \item Utilizar o modulo 're'.
\end{itemize}

%%%%%%%%%%%%%%%%%%%%%%
%%%%%Capitulo3%%%%%%%%
%%%%%%%%%%%%%%%%%%%%%%

\chapter{Concepção/desenho da Resolução}
\section{Organização}
Por forma a tornar a resolução do trabalho mais simples, decidimos criar uma função especifica para resolver cada uma das alineas do problema. Assim sendo, a solução do problema vai ser composta por 6 funções mais uma que vai servir de menu para o utilizador poder escolher qual das funcionalidades do prograna quer utilizar.\\
Na execução do programa todas as linhas do ficheiro 'clav-users.txt' são lidas para uma variavel que depois será utlizada por cada uma da funções.\\
Por uma questão de simplicidade, sempre que for necessário ordenar, a ordem utilizada é a a ordem alfabetica.
\section{Funcionalidades}
\subsection{Nome dos utilizadores ordenados e respetiva entidade.}
\subsection{Entidades ordenadas e número de utilizadores registos em cada uma.}
\subsection{Distribuiçao dos utilizadores por nivel de acesso.}
\subsection{Utilizadores agrupados por entidade, ordenados por nome e entidade.}
\subsection{Mostrar alguns indicadores.}
\subsection{Imprimir num ficheiro Json os 20 primeiros registos.}

%%%%%%%%%%%%%%%%%%%%%%
%%%%%Capitulo4%%%%%%%%
%%%%%%%%%%%%%%%%%%%%%%

\chapter{Demonstração de Funcionamento}
\section{Nome dos utilizadores ordenados e respetiva entidade.}
\section{Entidades ordenadas e número de utilizadores registos em cada uma.}
\section{Distribuiçao dos utilizadores por nivel de acesso.}
\section{Utilizadores agrupados por entidade, ordenados por nome e entidade.}
\section{Mostrar alguns indicadores.}
\section{Imprimir num ficheiro Json os 20 primeiros registos.}

%\VerbatimInput{teste1.txt}

%%%%%%%%%%%%%%%%%%%%%%
%%%%%Capitulo5%%%%%%%%
%%%%%%%%%%%%%%%%%%%%%%

\chapter{Conclusão} \label{concl}
Síntese do Documento~\cite{araujo:2018,martini:2018}.\\
Estado final do projecto; Análise crítica dos resultados~\cite{Sto77a}.\\
Trabalho futuro.

%%%%%%%%%%%%%%%%%%%%%%
%%%%%Capitulo6%%%%%%%%
%%%%%%%%%%%%%%%%%%%%%%

\appendix % apendice
\chapter{Código do Programa}

\begin{verbatim}
import re
import unidecode



fp = open('clav-users.txt', 'r')
text = fp.readlines()
fp.close()

def name_entity_list():
    result = []
    for line in text:
        user =  re.match(r'(\w+\s*(-?\w+\s*)*\b)',line).group()
        entity = re.search(r'ent_\w*',line).group()
        result.append((user, entity))

    result.sort(key=lambda x: unidecode.unidecode(x[0].casefold()))
    print("\n*** Utilizadores - Entidades ***\n")
    for r in result:
        print(r[0], "-", r[1])



def entity_num_elements_list():
    entities = {}
    for line in text:
        entity = re.search(r'ent_\w*', line).group()
        if entity not in entities:
            entities[entity] = 1
        else:
            entities[entity] +=1

    result = list(entities.items())
    result.sort()
    print("\n*** Entidades - Nº Utilizadores  ***\n")
    for r in result:
        print(r[0], "-", r[1])


def dist_users_level():
    levels = {}
    users = set()
    for line in text:
        broken_line = re.split(r'::', line)
        user = re.match(r'(\w+\s*(-?\w+\s*)*\b)', broken_line[0]).group()
        level = re.search(r'\d+\.?\d*',broken_line[3]).group()
        if level not in levels:
            levels[level] = set()
            levels[level].add(user)
        else:
            levels[level].add(user)
        users.add(user)
    result = list(levels.items())
    result.sort()
    total_users = len(users)
    print("\n*** Nivel de Acesso - Distribuição  ***\n")
    for r in result:
        print(f"Nivel {r[0]} - {round((len(r[1])/total_users)*100)}%" )
        for user in sorted(r[1], key=lambda x: unidecode.unidecode(x.casefold())):
            print(user)
        if(result.index(r) != len(result)-1):
            print("")


def name_entity_group():
    result = {}
    for line in text:
        name = re.match(r'(\w+\s*(-?\w+\s*)*\b)', line).group()
        entity = re.search(r'ent_\w*', line).group()
        if entity not in result:
            result[entity] = [name]
        else:
            result[entity].append(name)

    entities = list(result.keys())
    entities.sort()
    print("\n*** Utilizadores agrupados por entidade ***\n")
    for entity in entities:
        print(f"{entity}:")
        result[entity].sort(key=lambda x: unidecode.unidecode(x.casefold()))
        for user in result[entity]:
            print("*",user)
        print("")


def indicators():
    users = set()
    entities = {}
    levels = {}
    for line in text:
        broken_line = re.split(r'::', line)
        users.add(re.match(r'(\w+\s*(\w+\s*)*\b)',broken_line[0]).group())
        entity = re.search(r'ent_\w*', broken_line[2]).group()
        level = re.search(r'\d+\.?\d*', broken_line[3]).group()

        if entity not in entities:
            entities[entity] = 1
        else:
            entities[entity] += 1
        if level not in levels:
            levels[level] = 1
        else:
            levels[level] += 1

    print("\n*** Indicadores ***\n")
    print("")
    print("Número de Utilizadores:")
    print(len(users))
    print("")
    print("Número de Entidades:")
    print(len(entities.keys()))
    print("")
    print("Distribuição de utilizadores por entidade:")
    for entity in sorted(entities.keys()):
        print(f"* {entity} - {entities[entity]}")
    print("")
    print("Distribuição de utilizadores por nivel:")
    for level in sorted(levels.keys()):
        print(f"* {level} - {levels[level]}")


def json_20():
    list = []
    for i in range(20):
        broken_line = re.split(r'::', text[i])
        user = re.match(r'(\w+\s*(\w+\s*)*\b)', broken_line[0]).group()
        email = re.search(r'(\w+|\.|@|_|-)+', broken_line[1]).group()
        entity = re.search(r'ent_\w*', broken_line[2]).group()
        level = re.search(r'\d+\.?\d*', broken_line[3]).group()
        calls = re.search(r'\d+', broken_line[4]).group()
        list.append((user,email,entity,level,calls))

    file_name = input("\nDigite Nome Do Ficheiro de Output!\n>> ")
    fp = open(file_name, 'w')

    fp.write("{\n\t\"registos\":[")
    for i in range(len(list)):
        l = list[i]
        fp.write("\t\t{\n")
        fp.write(f"\t\t\t \"utilizador\":\"{l[0]}\",\n")
        fp.write(f"\t\t\t \"email\":\"{l[1]}\",\n")
        fp.write(f"\t\t\t \"entidade\":\"{l[2]}\",\n")
        fp.write(f"\t\t\t \"nivel de acesso\":\"{l[3]}\",\n")
        fp.write(f"\t\t\t \"número de chamadas ao backend\":\"{l[4]}\"\n")
        if i != 19:
            fp.write("\t\t},\n")
        else:
            fp.write("\t\t}\n")
    fp.write("\t]\n}\n")
    fp.close()
    print("\nFicheiro gerado com sucesso!")




def menu():
    cls = lambda: print('\n' * 50)
    inputfromuser = ""
    options = []
    options.append("Listagem com o nome e a entidade do utilizador, ordenada alfabeticamente por nome.")
    options.append("Lista ordenada alfabeticamente das entidades referenciadas, indicando, para cada uma, quantos utilizadores estão registados.")
    options.append("Distribuição de utilizadores por níveis de acesso.")
    options.append("Utilizadores, agrupados por entidade, ordenada primeiro pela entidade e dentro desta pelo nome.")
    options.append("Mostrar alguns indicadores.")
    options.append("Imprimir os 20 primeiros registos num novo ficheiro de output mas em formato jason.")
    cls()
    while inputfromuser != '0':
        print("*** Selecione Opção ***\n")
        for i in range(len(options)):
            print(f"{i+1}. {options[i]}")
        print("0. Sair.")

        inputfromuser = input(">> ")
        while not inputfromuser.isdigit() or int(inputfromuser) > len(options) or int(inputfromuser) < 0:
            print("Opçao Invalida!")
            inputfromuser = input(">> ")

        if inputfromuser == '0':
            continue
        elif inputfromuser == '1':
            name_entity_list()
        elif inputfromuser == '2':
            entity_num_elements_list()
        elif inputfromuser == '3':
            dist_users_level()
        elif inputfromuser == '4':
            name_entity_group()
        elif inputfromuser == '5':
            indicators()
        else:
            json_20()

        input("\nPressione Enter!\n>> ")
        cls()

menu()
\end{verbatim}




\lstinputlisting[caption={Exemplo de uma importação}, label={lstExe2}]{listagemImportadaLayout.l} %input de um ficheiro da listagem

%-- Fim do documento -- inserção das referencias bibliográficas

%\bibliographystyle{plain} % [1] Numérico pela ordem de citação ou ordem alfabetica
\bibliographystyle{alpha} % [Hen18] abreviação do apelido e data da publicação
%\bibliographystyle{apalike} % (Araujo, 2018) apelido e data da publicação
                             % --para usar este estilo descomente no inicio o comando \usepackage{apalike}

\bibliography{bibLayout} %input do ficheiro de referencias bibliograficas

\end{document} 