
\documentclass{letter} %especifica o tipo de documento que tenciona escrever: carta, artigo, relatório...

%\usepackage --influenciam o estilo de todo o documento
\usepackage[portuges]{babel}  %Babel -- irá activar automaticamente as regras apropriadas de hifenização para a língua
                              %portuges -- específica para o Português e outras línguas com caracteres acentuados.
\usepackage[utf8]{inputenc}

\setlength{\oddsidemargin}{-1cm}
\setlength{\textwidth}{18cm}
\setlength{\headsep}{-1cm}
\setlength{\textheight}{23cm}

\address{Pedro Rangel Henriques (gEPL)\\
         Departamento de Informática \\
         Universidade do Minho \\ Campus de Gualtar \\ 4700-057 Braga
} %Remetente

\signature{
  \begin{center}
         Pedro Henriques
  \end{center}
} %assinatura da carta

\date{24 Outubro de 2018}
%\date{\today}

\begin{document} % corpo do documento

\begin{letter}{Ex.mo Sr. Director de Curso\\
               Universidade do  Minho\\ Campus de Gualtar\\ Braga\\
               \vspace{1cm}
               \textbf{Assunto:} Disciplinas de Informática
} %destinatário e assunto

\opening{Caro Director de Curso} % abertura da carta

Desde o início, ................

Esperando ouvi-lo em breve, fico ao dispor para todos os
esclarecimentos que julgue convenientes.

\closing{Com os meus melhores cumprimentos,} % fecho da carta



\ps
P.S. Não se esqueça da reunião no proximo sábado.


\end{letter}

\end{document} 