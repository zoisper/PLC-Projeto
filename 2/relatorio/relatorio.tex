
\documentclass[11pt,a4paper]{report}%especifica o tipo de documento que tenciona escrever: carta, artigo, relatório... neste caso é um relatório
% [11pt,a4paper] Define o tamanho principal das letras do documento. caso não especifique uma delas, é assumido 10pt
% a4paper -- Define o tamanho do papel.

\usepackage[portuges]{babel}%Babel -- irá activar automaticamente as regras apropriadas de hifenização para a língua todo o
                                   %-- o texto gerado é automaticamente traduzido para Português.
                                   %  Por exemplo, “chapter” irá passar a “capítulo”, “table of contents” a “conteúdo”.
                                   % portuges -- específica para o Português.
\usepackage[utf8]{inputenc} % define o encoding usado texto fonte (input)--usual "utf8" ou "latin1

\usepackage{graphicx} %permite incluir graficos, tabelas, figuras
\usepackage{url} % para utilizar o comando \url{}
\usepackage{enumerate} %permite escolher, nas listas enumeradas, se os iems sao marcados com letras ou numeros-romanos em vez de numeracao normal

%\usepackage{apalike} % gerar biliografia no estilo 'named' (apalike)

\usepackage{color} % Para escrever em cores

\usepackage{multirow} %tabelas com multilinhas
\usepackage{array} %formatação especial de tabelas em array

\usepackage[pdftex]{hyperref} % transformar as referências internas do seu documento em hiper-ligações.

%Exemplos de fontes -- nao e vulgar mudar o tipo de fonte
%\usepackage{tgbonum} % Fonte de letra: TEX Gyre Bonum
%\usepackage{lmodern} % Fonte de letra: Latin Modern Sans Serif
%\usepackage{helvet}  % Fonte de letra: Helvetica
%\usepackage{charter} % Fonte de letra:Charter

\definecolor{saddlebrown}{rgb}{0.55, 0.27, 0.07} % para definir uma nova cor, neste caso 'saddlebrown'

\usepackage{listings}  % para utilizar blocos de texto verbatim no estilo 'listings'
%paramerização mais vulgar dos blocos LISTING - GENERAL
\lstset{
	basicstyle=\small, %o tamanho das fontes que são usadas para o código
	numbers=left, % onde colocar a numeração da linha
	numberstyle=\tiny, %o tamanho das fontes que são usadas para a numeração da linha
	numbersep=5pt, %distancia entre a numeração da linha e o codigo
	breaklines=true, %define quebra automática de linha
    frame=tB,  % caixa a volta do codigo
	mathescape=true, %habilita o modo matemático
	escapeinside={(*@}{@*)} % se escrever isto  aceita tudo o que esta dentro das marcas e nao altera
}
%
%\lstset{ %
%	language=Java,							% choose the language of the code
%	basicstyle=\ttfamily\footnotesize,		% the size of the fonts that are used for the code
%	keywordstyle=\bfseries,					% set the keyword style
%	%numbers=left,							% where to put the line-numbers
%	numberstyle=\scriptsize,				% the size of the fonts that are used for the line-numbers
%	stepnumber=2,							% the step between two line-numbers. If it's 1 each line
%											% will be numbered
%	numbersep=5pt,							% how far the line-numbers are from the code
%	backgroundcolor=\color{white},			% choose the background color. You must add \usepackage{color}
%	showspaces=false,						% show spaces adding particular underscores
%	showstringspaces=false,					% underline spaces within strings
%	showtabs=false,							% show tabs within strings adding particular underscores
%	frame=none,								% adds a frame around the code
%	%abovecaptionskip=-.8em,
%	%belowcaptionskip=.7em,
%	tabsize=2,								% sets default tabsize to 2 spaces
%	captionpos=b,							% sets the caption-position to bottom
%	breaklines=true,						% sets automatic line breaking
%	breakatwhitespace=false,				% sets if automatic breaks should only happen at whitespace
%	title=\lstname,							% show the filename of files included with \lstinputlisting;
%											% also try caption instead of title
%	escapeinside={\%*}{*)},					% if you want to add a comment within your code
%	morekeywords={*,...}					% if you want to add more keywords to the set
%}

\usepackage{xspace} % deteta se a seguir a palavra tem uma palavra ou um sinal de pontuaçao se tiver uma palavra da espaço, se for um sinal de pontuaçao nao da espaço

\parindent=0pt %espaço a deixar para fazer a  indentação da primeira linha após um parágrafo
\parskip=2pt % espaço entre o parágrafo e o texto anterior

\setlength{\oddsidemargin}{-1cm} %espaço entre o texto e a margem
\setlength{\textwidth}{18cm} %Comprimento do texto na pagina
\setlength{\headsep}{-1cm} %espaço entre o texto e o cabeçalho
\setlength{\textheight}{23cm} %altura do texto na pagina

% comando '\def' usado para definir abreviatura (macros)
% o primeiro argumento é o nome do novo comando e o segundo entre chavetas é o texto original, ou sequência de controle, para que expande
\def\darius{\textsf{Darius}\xspace}
\def\antlr{\texttt{AnTLR}\xspace}
\def\pe{\emph{Publicação Eletrónica}\xspace}
\def\titulo#1{\section{#1}}    %no corpo do documento usa-se na forma '\titulo{MEU TITULO}'
\def\super#1{{\em Supervisor: #1}\\ }
\def\area#1{{\em \'{A}rea: #1}\\[0.2cm]}
\def\resumo{\underline{Resumo}:\\ }

%\input{LPgeneralDefintions} %permite ler de um ficheiro de texto externo mais definições

\title{Processamento de Linguaguens e Compiladores (3º ano LCC)\\
       \textbf{Trabalho Prático 1}\\ Relatório de Desenvolvimento \\ \textbf{Grupo 17}\\ Problema 3
       } %Titulo do documento
%\title{Um Exemplo de Artigo em \LaTeX}
\author{José Pedro Gomes Ferreira\\ A91636 \\  \and Pedro Paulo Costa Pereira\\ A88062
         \and Tiago André Oliveira Leite\\ A91693
       } %autores do documento
\date{\today} %data

\begin{document} % corpo do documento
\maketitle % apresentar titulo, autor e data


\tableofcontents % Insere a tabela de indice
%\listoffigures % Insere a tabela de indice figuras
%\listoftables % Insere a tabela de indice tabelas


%%%%%%%%%%%%%%%%%%%%%%
%%%%%Capitulo1%%%%%%%%
%%%%%%%%%%%%%%%%%%%%%%


\chapter{Introdução} \label{chap:intro} %referência cruzada

Este documento tem como objetivo explicar a solução que implementamos para a resolução do probelma proposto no âmbito da unidade curricular de Processsamento de Linguagens e Compiladores. \\ \\
O problema consiste em implementar uma linguagem de programação imperativa simples, com regras sintaticas definidas pelo grupo. \\ \\ Para o desenvolvimento da nosso linguagem tivemos que definir uma gramática independente de contexto \textbf{GIC} e desenvolver um compilador que gera \textbf{pseudo-código}, \verb Assembly \ para uma Máquina Virtual, VM, que nos foi fornecida. \\ \\
No desenvolvimento do programa, para alem do conhecimento adquirido nas aulas, utilizamos os módulos 'Yacc/ Lex' do 'PLY/Python'. \\ \\ 
Esperamos que o resultado final cumpra todos os requisitos.  


%%%%%%%%%%%%%%%%%%%%%%
%%%%%Capitulo2%%%%%%%%
%%%%%%%%%%%%%%%%%%%%%%



\chapter{Problema Proposto} \label{chap:ProbelemaProposto} %capitulo e referencia cruzada

\section{Descrição} \label{sec:Descricao} %seccao e referencia cruzada
Pretende-se que comece por definir uma linguagem de programação imperativa simples, a seu gosto.
Apenas deve ter em consideração que essa linguagem terá de permitir:
\begin{itemize}

\item \textit{declarar} variáveis atómicas do tipo \textit{inteiro}, com os quais se podem realizar as habituais operações aritméticas, relacionais e lógicas;

\item \textit{efetuar} instruções algorı́tmicas básicas como a \textit{atribuição do valor de expressões numéricas a variáveis};

\item \textit{ler} do \textit{standard input} e \textit{escrever} no \textit{standard output};

\item \textit{efetuar} instruções \textit{condicionais} para controlo do fluxo de execução;

\item \textit{efetuar} instruções cı́clicas para controlo do fluxo de execução, permitindo o seu aninhamento.\\
\underline{Note} que deve implementar pelo menos o ciclo \textbf{while-do}, \textbf{repeat-until} ou \textbf{for-do}.
\end{itemize}

Adicionalmente deve ainda suportar, à sua escolha, uma das duas funcionalidades seguintes:
\begin{itemize}
\item \textit{declarar e manusear} variáveis estruturadas do tipo \textit{array (a 1 ou 2 dimensões)} de inteiros, em relação aos quais é apenas permitida a operação de indexação (ı́ndice inteiro);
\item \textit{definir e invocar subprogramas} sem parâmetros mas que possam retornar um resultado do tipo inteiro.
\end{itemize}


\section{Requisitos} \label{sec:Requesitos}
\begin{itemize}
  \item Utilização da linguagem Python.
  \item Resolver o problema recurso aos módulos 'Yacc/ Lex' do 'PLY/Python'.
  \item Gerar \textbf{pseudo-código}, \verb Assembly \ da Máquina Virtual VM fornecida.
  \item Preparar um conjunto de testes de modo a ver
o código Assembly gerado bem como o programa a correr na máquina virtual VM. Este conjunto terá de conter, no mı́nimo, os 4 primeiros exemplos abaixo e um dos 2 últimos conforme a escolha de funcionalidades da linguagem:
\begin{itemize}
\item ler 4 números e dizer se podem ser os lados de um quadrado;
\item ler um inteiro N, depois ler N números e escrever o menor deles;
\item ler N (constante do programa) números e calcular e imprimir o seu produtório;
\item contar e imprimir os números impares de uma sequência de números naturais;
\item ler e armazenar N números num array; imprimir os valores por ordem inversa;
\item invocar e usar num programa seu uma função ’potencia()’, que começa por ler do input a base $B$ e o expoente $E$ e retorna o valor  $B^E$.
\end{itemize}
\end{itemize}

%%%%%%%%%%%%%%%%%%%%%%
%%%%%Capitulo3%%%%%%%%
%%%%%%%%%%%%%%%%%%%%%%

\chapter{Concepção/desenho da Resolução}
\section{Organização e estrutura}
O nosso trabalho pode ser divido em \textbf{4} partes:
\begin{itemize}
\item Definição da \textbf{GIC} que define a estrura sintática da nosso liguaguem.
\item Construção do analizador léxico \verb lexer .
\item Construção do analizador sintático \verb parser .
\item Conversão das instruções para código \verb Assembly \ da  VM.    
\end{itemize}
Todas as funcionalidades descritas neste capítulo podem ser encontradas no anexo A do documento.\\
No nosso trabalho optamos por implementar a funcionalidade de textit{declarar e manusear} variáveis estruturadas do tipo \textit{array (a 1 ou 2 dimensões)} de inteiros, em relação aos quais é apenas permitida a operação de indexação (ı́ndice inteiro).\\

\section{GIC}
A nossa linguagem é gerada pela seguinte grámatica independente de contexto:

\begin{verbatim}
program      : MAIN LCURLY body RCURLY

body         : declarations instructions

declarations : 
             | declaration  declarations

declaration  : type VAR SEMICOLON
             | type LBRACKET NUM RBRACKET VAR SEMICOLON
             | type LBRACKET NUM RBRACKET LBRACKET NUM RBRACKET VAR SEMICOLON

type         : INT
             | FLOAT

instructions :
             | instruction  instructions

instruction  : atributions SEMICOLON
             | WHILE LPAREN condition RPAREN LCURLY instructions RCURLY
             | FOR LPAREN atributions SEMICOLON condition SEMICOLON atributions RPAREN LCURLY 
               instructions RCURLY
             | IF LPAREN condition RPAREN LCURLY instructions RCURLY
             | IF LPAREN condition RPAREN LCURLY instructions RCURLY ELSE LCURLY 
               instructions RCURLY
             | SCAN LPAREN variable RPAREN SEMICOLON
             | PRINT LPAREN variable RPAREN SEMICOLON
             | PRINTLN LPAREN variable RPAREN SEMICOLON
             | PRINT LPAREN STRING RPAREN SEMICOLON
             | PRINTLN LPAREN STRING RPAREN SEMICOLON

atributions  : 
             | atribution
             | atribution COMMA atributions

atribution   : variable EQUAL expression 
             | variable EQUAL condition

variable     : VAR
             | VAR LBRACKET expression RBRACKET
             | VAR LBRACKET expression RBRACKET LBRACKET expression RBRACKET

expression   : variable
             | NUM
             | REAL
             | LPAREN expression RPAREN
             | expression PLUS expression
             | expression MINUS expression
             | expression MUL expression
             | expression DIV expression
             | expression MOD expression

condition    : expression EQEQ expression
             | expression DIFF expression
             | expression GREATER expression
             | expression LESSER expression
             | expression GREATEQ expression
             | expression LESSEQ expression
             | NUM
             | REAL
             | variable
\end{verbatim}


\section{Lexer}
O analisador léxico, \verb lexer, vai ser o responsavel por 'capturar' os simbolos terminais, \verb tokens, da nossa linguagem através de expressões regulares. Para a implementação do analisador léxico utilizamos o módulo 'Lex' do 'PLY/Python'.\\
De seguida apresentamos os \verb tokens \ da nossa linguagem e respetivas expressão regular.
\begin{verbatim}

SEMICOLON : ';'
COMMA     : ','
LCURLY    : '\{'
RCURLY    : '\}'
LPAREN    : '\('
RPAREN    : '\)'
LBRACKET  : '\['
RBRACKET  : '\]'
FLOAT     : 'float'
INT       : 'int'
MAIN      : 'main'
WHILE     : 'while'
FOR       : 'for'
IF        : 'if'
ELSE      : 'else'
PRINTLN   : 'println'
PRINT     : 'print'
SCAN      : 'scan'
STRING    : '"([^"]|(\\n))*"'
REAL      : '-?([1-9][0-9]*\.[0-9]+|0\.[0-9]+)'
NUM       : '-?\d+'
EQUAL     : '\=\='
DIFF      : '\!\='
GREATEQ   : '\>\='
LESSEQ    : '\<\='
GREATER   : '\>'
LESSER    : '\<'
EQEQ      : '\='
PLUS      : '\+'
MINUS     : '\-'
MUL       : '\*'
DIV       : '\/'
MOD       : '\%'
VAR       : '\w+'
\end{verbatim}
A implementação do analisador léxico ser encontrada no anexo A do documento.   

\section{Parser e geração do codigo Assembly da VM}
O analisador sintático, \verb parser, vai ser o responsavel por verificar se o código que foi escrito na nossa linguagem está sintaticamente correto, isto é, respeitas as regras gramaticais definidas. Caso não haja erros sintáticos o \parser converte o código da nossa linguagem em codigo \verb Assembly \ da máquina virtual. Se houver erros, é mostrado ao utilizador uma mensagem de erro sitático.
A implementação do analisador sintático pode ser encontrada no anexo A do documento. 


%%%%%%%%%%%%%%%%%%%%%%
%%%%%Capitulo4%%%%%%%%
%%%%%%%%%%%%%%%%%%%%%%

\chapter{Demonstração de Funcionamento}
\section{Geração e execução de código Assembly}
Para utilizar a nossa linguagem, o utilizador tem que:
\begin{itemize}
\item Escrever e guardar as instruções num ficheiro de texto de acordo com as regras grámaticais da linguagem. 
\item Colocar o ficheiro na mesma diretoria dos ficheiros \verb lexer.py, \verb parser/py \verb vms \ e \verb vmsGTKAux. 
\item Executar um dos seguintes comandos: 
\begin{verbatim} 
>> python3 parser.py <ficheiro de input> 
>> python3 parser.py <ficheiro de input> <ficheiro de output>
\end{verbatim} 
\underline{Nota:} Caso o utilizador opte pelo primeiro comando, vai ser criado um ficheiro denominado
\verb a.vm  onde será colocado o código gerado.
\item Executar o comando:
\begin{verbatim} 
>> ./vms <ficheiro gerado no comando anterior> 
\end{verbatim} 
\end{itemize}


\section{Teste 1}
Ler 4 números e dizer se podem ser os lados de um quadrado. \\
\underline{Ficheiro de input:} 'quadrado.txt'. 

\subsection{Conteúdo do ficheiro}

\begin{verbatim}
main
{
  float[4] lados;
  int[6] a;
  int i;
  i = 4;
  
  while(i){
    print("Digite lado: ");
    scan(lados[4 - i]);
    i = i - 1;  
  }
  
  a[0] = lados[0] == lados[1];
  a[1] = lados[0] == lados[2];
  a[2] = lados[0] == lados[3];
  a[3] = lados[1] == lados[2];
  a[4] = lados[1] == lados[3];
  a[5] = lados[2] == lados[3];
  
  
  i = a[0]*a[1]*a[2]*a[3]*a[4]*a[5];
  
  if(i)
  {  
    
    if(lados[0])
    {
      println("Podem ser os lados de um quadrado!");

    }

    else
    {
      println("Não podem ser os lados de um quadrado!");
    }
  }

  else
  {
    println("Não podem ser os lados de um quadrado!");
  }

}
\end{verbatim}


\subsection{Código assembly gerado}
\begin{verbatim}
PUSHN 4
PUSHN 6
PUSHI 0
START
PUSHGP
PUSHI 10
PUSHI 4
STOREN
B0:
PUSHGP
PUSHI 10
LOADN
PUSHI 0
SUP
JZ E0
PUSHS "Digite lado: "
WRITES
PUSHGP
PUSHI 0
PUSHI 4
PUSHGP
PUSHI 10
LOADN
SUB
ADD
READ
ATOF
STOREN
PUSHGP
PUSHI 10
PUSHGP
PUSHI 10
LOADN
PUSHI 1
SUB
STOREN
JUMP B0
E0:
PUSHGP
PUSHI 4
PUSHI 0
ADD
PUSHGP
PUSHI 0
PUSHI 0
ADD
LOADN
PUSHGP
PUSHI 0
PUSHI 1
ADD
LOADN
EQUAL
STOREN
PUSHGP
PUSHI 4
PUSHI 1
ADD
PUSHGP
PUSHI 0
PUSHI 0
ADD
LOADN
PUSHGP
PUSHI 0
PUSHI 2
ADD
LOADN
EQUAL
STOREN
PUSHGP
PUSHI 4
PUSHI 2
ADD
PUSHGP
PUSHI 0
PUSHI 0
ADD
LOADN
PUSHGP
PUSHI 0
PUSHI 3
ADD
LOADN
EQUAL
STOREN
PUSHGP
PUSHI 4
PUSHI 3
ADD
PUSHGP
PUSHI 0
PUSHI 1
ADD
LOADN
PUSHGP
PUSHI 0
PUSHI 2
ADD
LOADN
EQUAL
STOREN
PUSHGP
PUSHI 4
PUSHI 4
ADD
PUSHGP
PUSHI 0
PUSHI 1
ADD
LOADN
PUSHGP
PUSHI 0
PUSHI 3
ADD
LOADN
EQUAL
STOREN
PUSHGP
PUSHI 4
PUSHI 5
ADD
PUSHGP
PUSHI 0
PUSHI 2
ADD
LOADN
PUSHGP
PUSHI 0
PUSHI 3
ADD
LOADN
EQUAL
STOREN
PUSHGP
PUSHI 10
PUSHGP
PUSHI 4
PUSHI 0
ADD
LOADN
PUSHGP
PUSHI 4
PUSHI 1
ADD
LOADN
MUL
PUSHGP
PUSHI 4
PUSHI 2
ADD
LOADN
MUL
PUSHGP
PUSHI 4
PUSHI 3
ADD
LOADN
MUL
PUSHGP
PUSHI 4
PUSHI 4
ADD
LOADN
MUL
PUSHGP
PUSHI 4
PUSHI 5
ADD
LOADN
MUL
STOREN
PUSHGP
PUSHI 10
LOADN
PUSHI 0
SUP
JZ E2
PUSHGP
PUSHI 0
PUSHI 0
ADD
LOADN
PUSHF 0.0
FSUP
FTOI
PUSHI 0
SUP
JZ E1
PUSHS "Podem ser os lados de um quadrado!"
WRITES
PUSHS"\n"
WRITES
JUMP F1
E1:
PUSHS "Não podem ser os lados de um quadrado!"
WRITES
PUSHS"\n"
WRITES
F1:
JUMP F2
E2:
PUSHS "Não podem ser os lados de um quadrado!"
WRITES
PUSHS"\n"
WRITES
F2:
STOP
\end{verbatim}


\subsection{Execução da VM com o código gerado}

\begin{verbatim}
>> ./vms a.vm
Digite lado: 1
Digite lado: 1
Digite lado: 1
Digite lado: 1
Podem ser os lados de um quadrado!
>>
\end{verbatim}

\begin{verbatim}
>> ./vms a.vm
Digite lado: 0
Digite lado: 0
Digite lado: 0
Digite lado: 0
Não podem ser os lados de um quadrado!
>>
\end{verbatim}

\begin{verbatim}
>> ./vms a.vm
Digite lado: 0.1
Digite lado: 0.1
Digite lado: 0.1
Digite lado: 0.1
Podem ser os lados de um quadrado!
\end{verbatim}

\begin{verbatim}
>> ./vms a.vm
Digite lado: 4
Digite lado: 4
Digite lado: 6
Digite lado: 4
Não podem ser os lados de um quadrado!
>>
\end{verbatim}


\section{Teste 2}
Ler um inteiro N, depois ler N números e escrever o menor deles. \\
\underline{Ficheiro de input:} 'menor.txt'.

\subsection{Conteúdo do ficheiro}

\begin{verbatim}
main
{
  
  int menor;
  int N;
  int i;
  int aux;
  
  i = 0;
  
  print("Digite número: ");
  scan(menor);
  

  print("Digite quantos números quer ler: ");
  scan(N);


  while(i<N){
    print("Digite número: ");
    scan(aux);

    if(aux < menor){
      menor = aux;
    }
    i = i +1;
  }
  print("O menor número é: ");
  println(menor);

}
\end{verbatim}


\subsection{Código assembly gerado}
\begin{verbatim}
PUSHI 0
PUSHI 0
PUSHI 0
PUSHI 0
START
PUSHGP
PUSHI 2
PUSHI 0
STOREN
PUSHS "Digite número: "
WRITES
PUSHGP
PUSHI 0
READ
ATOI
STOREN
PUSHS "Digite quantos números quer ler: "
WRITES
PUSHGP
PUSHI 1
READ
ATOI
STOREN
B1:
PUSHGP
PUSHI 2
LOADN
PUSHGP
PUSHI 1
LOADN
INF
JZ E1
PUSHS "Digite número: "
WRITES
PUSHGP
PUSHI 3
READ
ATOI
STOREN
PUSHGP
PUSHI 3
LOADN
PUSHGP
PUSHI 0
LOADN
INF
JZ E0
PUSHGP
PUSHI 0
PUSHGP
PUSHI 3
LOADN
STOREN
E0:
PUSHGP
PUSHI 2
PUSHGP
PUSHI 2
LOADN
PUSHI 1
ADD
STOREN
JUMP B1
E1:
PUSHS "O menor número é: "
WRITES
PUSHGP
PUSHI 0
LOADN
WRITEI
PUSHS"\n"
WRITES
STOP
\end{verbatim}


\subsection{Execução da VM com o código gerado}

\begin{verbatim}
>> ./vms a.vm
Digite número: 5
Digite quantos números quer ler: 3
Digite número: -1
Digite número: 4
Digite número: 0
O menor número é: -1
>>
\end{verbatim}

\begin{verbatim}
>> ./vms a.vm
Digite número: 2
Digite quantos números quer ler: 0
O menor número é: 2
>>
\end{verbatim}

\begin{verbatim}
>> ./vms a.vm
Digite número: 1
Digite quantos números quer ler: 5
Digite número: 2
Digite número: 3
Digite número: 4
Digite número: 5
Digite número: 0
O menor número é: 0
\end{verbatim}

\section{Teste 3}
Ler N (constante do programa) números e calcular e imprimir o seu produtório. \\
\underline{Ficheiro de input:} 'produto.txt'.

\subsection{Conteúdo do ficheiro}

\begin{verbatim}
main
{
  
  int N;
  int r;
  int[5] a;
  int i;

  i = 0;
  N = 5;
  r = 1;
  
  while(i<N){
    print("Digite um número: ");
    scan(a[i]);
    i = i + 1;

  }
  i = 0;

  while(i<N){
    print(a[i]);
    print(" x ");
    r = r * a[i];
    i = i + 1;
  }
  
  print(" = ");
  println(r);

}
\end{verbatim}


\subsection{Código assembly gerado}
\begin{verbatim}
PUSHI 0
PUSHI 0
PUSHN 5
PUSHI 0
START
PUSHGP
PUSHI 7
PUSHI 0
STOREN
PUSHGP
PUSHI 0
PUSHI 5
STOREN
PUSHGP
PUSHI 1
PUSHI 1
STOREN
B0:
PUSHGP
PUSHI 7
LOADN
PUSHGP
PUSHI 0
LOADN
INF
JZ E0
PUSHS "Digite um número: "
WRITES
PUSHGP
PUSHI 2
PUSHGP
PUSHI 7
LOADN
ADD
READ
ATOI
STOREN
PUSHGP
PUSHI 7
PUSHGP
PUSHI 7
LOADN
PUSHI 1
ADD
STOREN
JUMP B0
E0:
PUSHGP
PUSHI 7
PUSHI 0
STOREN
B1:
PUSHGP
PUSHI 7
LOADN
PUSHGP
PUSHI 0
LOADN
INF
JZ E1
PUSHGP
PUSHI 2
PUSHGP
PUSHI 7
LOADN
ADD
LOADN
WRITEI
PUSHS " x "
WRITES
PUSHGP
PUSHI 1
PUSHGP
PUSHI 1
LOADN
PUSHGP
PUSHI 2
PUSHGP
PUSHI 7
LOADN
ADD
LOADN
MUL
STOREN
PUSHGP
PUSHI 7
PUSHGP
PUSHI 7
LOADN
PUSHI 1
ADD
STOREN
JUMP B1
E1:
PUSHS " = "
WRITES
PUSHGP
PUSHI 1
LOADN
WRITEI
PUSHS"\n"
WRITES
STOP
\end{verbatim}


\subsection{Execução da VM com o código gerado}

\begin{verbatim}
>> ./vms a.vm
Digite um número: 1
Digite um número: 2
Digite um número: 3
Digite um número: 4
Digite um número: 5
1 x 2 x 3 x 4 x 5 x  = 120
>>
\end{verbatim}

\begin{verbatim}
>> ./vms a.vm
Digite um número: 0
Digite um número: 1
Digite um número: 2
Digite um número: 3
Digite um número: 4
0 x 1 x 2 x 3 x 4 x  = 0
>>
\end{verbatim}

\begin{verbatim}
>> ./vms a.vm
Digite um número: 1.5
Digite um número: 1.5
Digite um número: 1
Digite um número: 2
Digite um número: 3
1 x 1 x 1 x 2 x 3 x  = 6
>>
\end{verbatim}

\section{Teste 4}
Contar e imprimir os números impares de uma sequência de números naturais. \\
\underline{Ficheiro de input:} 'impares.txt'.

\subsection{Conteúdo do ficheiro}

\begin{verbatim}
main
{
  
  int count;
  int aux;
  count = 0;
  aux = 1;
  println("Digite 0 para parar!");
  while(aux !=0){
    print("Digite número: ");
    scan(aux);
    if(aux %2 == 1){
      print(aux);
      println(" é impar!");
      count = count +1;
    }
  }
  print("Foram lidos ");
  print(count);
  println(" números impares!");

}
\end{verbatim}


\subsection{Código assembly gerado}
\begin{verbatim}
PUSHI 0
PUSHI 0
START
PUSHGP
PUSHI 0
PUSHI 0
STOREN
PUSHGP
PUSHI 1
PUSHI 1
STOREN
PUSHS "Digite 0 para parar!"
WRITES
PUSHS"\n"
WRITES
B1:
PUSHGP
PUSHI 1
LOADN
PUSHI 0
EQUAL
NOT
JZ E1
PUSHS "Digite número: "
WRITES
PUSHGP
PUSHI 1
READ
ATOI
STOREN
PUSHGP
PUSHI 1
LOADN
PUSHI 2
MOD
PUSHI 1
EQUAL
JZ E0
PUSHGP
PUSHI 1
LOADN
WRITEI
PUSHS " é impar!"
WRITES
PUSHS"\n"
WRITES
PUSHGP
PUSHI 0
PUSHGP
PUSHI 0
LOADN
PUSHI 1
ADD
STOREN
E0:
JUMP B1
E1:
PUSHS "Foram lidos "
WRITES
PUSHGP
PUSHI 0
LOADN
WRITEI
PUSHS " números impares!"
WRITES
PUSHS"\n"
WRITES
STOP
\end{verbatim}


\subsection{Execução da VM com o código gerado}

\begin{verbatim}
>> ./vms a.vm
Digite 0 para parar!
Digite número: 1
1 é impar!
Digite número: 2
Digite número: 3
3 é impar!
Digite número: 4
Digite número: 5
5 é impar!
Digite número: 6
Digite número: 0
Foram lidos 3 números impares!
>>
\end{verbatim}

\section{Teste 5}
Ler e armazenar N números num array. Imprimir os valores por ordem inversa. \\
\underline{Ficheiro de input:} 'inversa.txt'.

\subsection{Conteúdo do ficheiro}

\begin{verbatim}
main
{
  int N;
  int i;
  int[10] a;
  N = 5;
  i = 0;

  print("Neste programa vamos digitar ");
  print(N);
  println(" números e imprimi-los por ordem inversa.");

  while(i<N){
    print("Digite número: ");
    scan(a[i]); 
    i = i +1;
  }

  i = N - 1;

  while(i>=0){
    print(a[i]);
    print(" ");
    i = i - 1;
  }

  println("");

}
\end{verbatim}


\subsection{Código assembly gerado}
\begin{verbatim}
PUSHI 0
PUSHI 0
PUSHN 10
START
PUSHGP
PUSHI 0
PUSHI 5
STOREN
PUSHGP
PUSHI 1
PUSHI 0
STOREN
PUSHS "Neste programa vamos digitar "
WRITES
PUSHGP
PUSHI 0
LOADN
WRITEI
PUSHS " números e imprimi-los por ordem inversa."
WRITES
PUSHS"\n"
WRITES
B0:
PUSHGP
PUSHI 1
LOADN
PUSHGP
PUSHI 0
LOADN
INF
JZ E0
PUSHS "Digite número: "
WRITES
PUSHGP
PUSHI 2
PUSHGP
PUSHI 1
LOADN
ADD
READ
ATOI
STOREN
PUSHGP
PUSHI 1
PUSHGP
PUSHI 1
LOADN
PUSHI 1
ADD
STOREN
JUMP B0
E0:
PUSHGP
PUSHI 1
PUSHGP
PUSHI 0
LOADN
PUSHI 1
SUB
STOREN
B1:
PUSHGP
PUSHI 1
LOADN
PUSHI 0
SUPEQ
JZ E1
PUSHGP
PUSHI 2
PUSHGP
PUSHI 1
LOADN
ADD
LOADN
WRITEI
PUSHS " "
WRITES
PUSHGP
PUSHI 1
PUSHGP
PUSHI 1
LOADN
PUSHI 1
SUB
STOREN
JUMP B1
E1:
PUSHS ""
WRITES
PUSHS"\n"
WRITES
STOP
\end{verbatim}


\subsection{Execução da VM com o código gerado}

\begin{verbatim}
>> ./vms a.vm
Neste programa vamos digitar 5 números e imprimi-los por ordem inversa.
Digite número: 1
Digite número: 2
Digite número: 3
Digite número: 4
Digite número: 5
5 4 3 2 1
>>
\end{verbatim}

\begin{verbatim}
>> ./vms a.vm
Neste programa vamos digitar 5 números e imprimi-los por ordem inversa.
Digite número: 3
Digite número: 4
Digite número: 1
Digite número: 7
Digite número: 6
6 7 1 4 3
>>
\end{verbatim}

\section{Teste 6}
Execução de operações aritméticas. \\
\underline{Ficheiro de input:} 'calculo.txt'.

\subsection{Conteúdo do ficheiro}

\begin{verbatim}
main
{
  int a;


  a = 2 * 3 + 4 * 5 - 1;
  
  print("a = 2 * 3 + 4 * 5 - 1 = ");
  println(a);
  

  a = (1 + a) * a;
  
  print("(1 + a) * a = ");
  println(a); 

}
\end{verbatim}


\subsection{Código assembly gerado}
\begin{verbatim}
PUSHI 0
START
PUSHGP
PUSHI 0
PUSHI 2
PUSHI 3
MUL
PUSHI 4
PUSHI 5
MUL
ADD
PUSHI 1
SUB
STOREN
PUSHS "a = 2 * 3 + 4 * 5 - 1 = "
WRITES
PUSHGP
PUSHI 0
LOADN
WRITEI
PUSHS"\n"
WRITES
PUSHGP
PUSHI 0
PUSHI 1
PUSHGP
PUSHI 0
LOADN
ADD
PUSHGP
PUSHI 0
LOADN
MUL
STOREN
PUSHS "(1 + a) * a = "
WRITES
PUSHGP
PUSHI 0
LOADN
WRITEI
PUSHS"\n"
WRITES
STOP
\end{verbatim}


\subsection{Execução da VM com o código gerado}

\begin{verbatim}
>> ./vms a.vm
a = 2 * 3 + 4 * 5 - 1 = 25
(1 + a) * a = 650
>>
\end{verbatim}

\section{Teste 7}
Ordenação de um array. \\
\underline{Ficheiro de input:} 'ordena.txt'.

\subsection{Conteúdo do ficheiro}

\begin{verbatim}
main
{
  int N;
  int i;
  int j;
  int menor;
  int aux;
  int[5] a;
  

  N = 5;

  print("Neste programa vamos digitar ");
  print(N);
  println(" números e imprimi-los por ordem crescente.");

  for(i=0;i<N;i=i+1){
    print("Digite número: ");
    scan(a[i]); 
  }
  

  for(i=0;i<N;i=i+1)
  {
    menor = i;

    for(j=i+1;j<N;j=j+1)
    {
      if(a[j]<a[menor])
      {
        menor = j;
      }
    }
    
    if(menor !=i)
    {
      aux = a[i];
      a[i] = a[menor];
      a[menor] = aux;

    }



  } 



  for(i=0;i<N; i=i+1)
  {
    print(a[i]);
    print(" ");
  }

  println("");
  
}
\end{verbatim}


\subsection{Código assembly gerado}
\begin{verbatim}
PUSHI 0
PUSHI 0
PUSHI 0
PUSHI 0
PUSHI 0
PUSHN 5
START
PUSHGP
PUSHI 0
PUSHI 5
STOREN
PUSHS "Neste programa vamos digitar "
WRITES
PUSHGP
PUSHI 0
LOADN
WRITEI
PUSHS " números e imprimi-los por ordem crescente."
WRITES
PUSHS"\n"
WRITES
PUSHGP
PUSHI 1
PUSHI 0
STOREN
B0:
PUSHGP
PUSHI 1
LOADN
PUSHGP
PUSHI 0
LOADN
INF
JZ E0
PUSHS "Digite número: "
WRITES
PUSHGP
PUSHI 5
PUSHGP
PUSHI 1
LOADN
ADD
READ
ATOI
STOREN
PUSHGP
PUSHI 1
PUSHGP
PUSHI 1
LOADN
PUSHI 1
ADD
STOREN
JUMP B0
E0:
PUSHGP
PUSHI 1
PUSHI 0
STOREN
B4:
PUSHGP
PUSHI 1
LOADN
PUSHGP
PUSHI 0
LOADN
INF
JZ E4
PUSHGP
PUSHI 3
PUSHGP
PUSHI 1
LOADN
STOREN
PUSHGP
PUSHI 2
PUSHGP
PUSHI 1
LOADN
PUSHI 1
ADD
STOREN
B2:
PUSHGP
PUSHI 2
LOADN
PUSHGP
PUSHI 0
LOADN
INF
JZ E2
PUSHGP
PUSHI 5
PUSHGP
PUSHI 2
LOADN
ADD
LOADN
PUSHGP
PUSHI 5
PUSHGP
PUSHI 3
LOADN
ADD
LOADN
INF
JZ E1
PUSHGP
PUSHI 3
PUSHGP
PUSHI 2
LOADN
STOREN
E1:
PUSHGP
PUSHI 2
PUSHGP
PUSHI 2
LOADN
PUSHI 1
ADD
STOREN
JUMP B2
E2:
PUSHGP
PUSHI 3
LOADN
PUSHGP
PUSHI 1
LOADN
EQUAL
NOT
JZ E3
PUSHGP
PUSHI 4
PUSHGP
PUSHI 5
PUSHGP
PUSHI 1
LOADN
ADD
LOADN
STOREN
PUSHGP
PUSHI 5
PUSHGP
PUSHI 1
LOADN
ADD
PUSHGP
PUSHI 5
PUSHGP
PUSHI 3
LOADN
ADD
LOADN
STOREN
PUSHGP
PUSHI 5
PUSHGP
PUSHI 3
LOADN
ADD
PUSHGP
PUSHI 4
LOADN
STOREN
E3:
PUSHGP
PUSHI 1
PUSHGP
PUSHI 1
LOADN
PUSHI 1
ADD
STOREN
JUMP B4
E4:
PUSHGP
PUSHI 1
PUSHI 0
STOREN
B5:
PUSHGP
PUSHI 1
LOADN
PUSHGP
PUSHI 0
LOADN
INF
JZ E5
PUSHGP
PUSHI 5
PUSHGP
PUSHI 1
LOADN
ADD
LOADN
WRITEI
PUSHS " "
WRITES
PUSHGP
PUSHI 1
PUSHGP
PUSHI 1
LOADN
PUSHI 1
ADD
STOREN
JUMP B5
E5:
PUSHS ""
WRITES
PUSHS"\n"
WRITES
STOP
\end{verbatim}


\subsection{Execução da VM com o código gerado}

\begin{verbatim}
>> ./vms a.vm
Neste programa vamos digitar 5 números e imprimi-los por ordem crescente.
Digite número: 1
Digite número: 2
Digite número: 3
Digite número: 4
Digite número: 5
1 2 3 4 5 
>>
\end{verbatim}

\begin{verbatim}
>> ./vms a.vm
Neste programa vamos digitar 5 números e imprimi-los por ordem crescente.
Digite número: 5
Digite número: 4
Digite número: 3
Digite número: 2
Digite número: 1
1 2 3 4 5
>>
\end{verbatim}

\begin{verbatim}
>> ./vms a.vm
Neste programa vamos digitar 5 números e imprimi-los por ordem crescente.
Digite número: -1
Digite número: -5
Digite número: 3
Digite número: 0
Digite número: 3
-5 -1 0 3 3 
>>
\end{verbatim}  

%%%%%%%%%%%%%%%%%%%%%%
%%%%%Capitulo5%%%%%%%%
%%%%%%%%%%%%%%%%%%%%%%

\chapter{Conclusão} \label{concl}
Com o projeto concluído esperamos ter cumprido todos os requisitos que nos foram propostos. \\ \\
O facto de produzirmos a nossa própria linguagem tornou a experiencia bastante interessante, apesar de a sintaxe escolhida para a linguagem desenvolvida se aproximar muito da linguagem \verb C.  \\ \\
Achamos que há aspetos que poderiam ser melhorados, como por exemplo a implementação de funções com e sem argumentos. Também na parte de atribuição de valores as variaveis há uma pequena falha que permite ao utilizador por exemplo fazer:
\begin{verbatim}
a = 1,;
\end{verbatim}
No entanto se esta fosse removida, nao poderiamos por exemplo fazer a seguinte instrução:
\begin{verbatim}
a = 1,b = 2;
\end{verbatim}
E os ciclos \textit{for} teriam que ficar do genero:
\begin{verbatim}
for(i=0;i<N;i=i+i;){}
\end{verbatim}
Ou seja, a operação de incremento do \verb 'i' \ teria que ter um \verb ';' \ desnecessário.\\

Contudo, como temos a liberdade de definir a sintaxe da linguagem, podemos dizer que isto nao é um erro mas sim uma opção.\\

Por fim, tal como ja tinha sido mencionado no relatório do primeiro projeto, todos concordamos que o facto de o projecto ter sido desenvolvido na linguagem 'Python' e, neste caso, com recurso aos módulos 'Yacc/ Lex' do 'PLY/Python', facilitou bastante o seu desenvolvimento.

%%%%%%%%%%%%%%%%%%%%%%
%%%%%Capitulo6%%%%%%%%
%%%%%%%%%%%%%%%%%%%%%%

\appendix % apendice
\chapter{Código do Programa}

\textbf{Ficheiro lexer.py}
\begin{scriptsize}
\begin{verbatim}
import ply.lex as lex
import sys

tokens = ('LCURLY','RCURLY','LPAREN','RPAREN','LBRACKET','RBRACKET','NUM','REAL',
      'VAR','FLOAT','INT','SEMICOLON','COMMA','MAIN','WHILE','FOR','IF','ELSE',
      'STRING','EQUAL','PLUS','MINUS','MUL','DIV','MOD','EQEQ','DIFF','GREATER',
      'LESSER','GREATEQ','LESSEQ','SCAN','PRINT','PRINTLN')

def t_SEMICOLON(t):
  r';'
  return t

def t_COMMA(t):
  r','
  return t

def t_LCURLY(t):
  r'\{'
  return t

def t_RCURLY(t):
  r'\}'
  return t

def t_LPAREN(t):
  r'\('
  return t

def t_RPAREN(t):
  r'\)'
  return t

def t_LBRACKET(t):
  r'\['
  return t

def t_RBRACKET(t):
  r'\]'
  return t

def t_FLOAT(t):
  r'float'
  return t

def t_INT(t):
  r'int'
  return t

def t_MAIN(t):
  r'main'
  return t

def t_WHILE(t):
  r'while'
  return t

def t_FOR(t):
  r'for'
  return t

def t_IF(t):
  r'if'
  return t

def t_ELSE(t):
  r'else'
  return t

def t_PRINTLN(t):
  r'println'
  return t

def t_PRINT(t):
  r'print'
  return t

def t_SCAN(t):
  r'scan'
  return t

def t_STRING(t):
  r'"([^"]|(\\n))*"'
  return t

def t_REAL(t):
    r'-?([1-9][0-9]*\.[0-9]+|0\.[0-9]+)'
    return t

def t_NUM(t):
  r'-?\d+'
  t.value = int(t.value)
  return t

def t_EQEQ(t):
  r'\=\='
  return t

def t_DIFF(t):
  r'\!\='
  return t

def t_GREATEQ(t):
  r'\>\='
  return t

def t_LESSEQ(t):
  r'\<\='
  return t

def t_GREATER(t):
  r'\>'
  return t

def t_LESSER(t):
  r'\<'
  return t

def t_EQUAL(t):
  r'\='
  return t

def t_PLUS(t):
  r'\+'
  return t

def t_MINUS(t):
  r'\-'
  return t

def t_MUL(t):
  r'\*'
  return t

def t_DIV(t):
  r'\/'
  return t

def t_MOD(t):
  r'\%'
  return t

def t_VAR(t):
  r'\w+'
  return t

def t_error(t):
  print("Illegal Character:", t.value[0])
  t.lexer.skip(1)

t_ignore = ' \r\n\t'

lexer = lex.lex()

\end{verbatim}
\end{scriptsize}

\textbf{Ficheiro parser.py}
\begin{scriptsize}
\begin{verbatim}
import ply.yacc as yacc
import sys
import os.path
from lexer import tokens

precedence = (
    ('left', 'PLUS', 'MINUS'),
    ('left', 'MUL', 'DIV'),
)


def var_new(v):
  if v[0] in parser.tab_id or v[1] == 0 or v[2] == 0:
    return -1
  else:
    parser.tab_id[v[0]] = (parser.prox_address,v[1],v[2],v[3])
    parser.prox_address += v[1]*v[2]
    return 0


def var_address_base(v):
  if v in parser.tab_id:
    return parser.tab_id[v][0]
  else:
    return -1


def var_num_colums(v):
  if v in parser.tab_id:
    return parser.tab_id[v][1]
  else:
    return -1


def var_num_lines(v):
  if v in parser.tab_id:
    return parser.tab_id[v][2]
  else:
    return -1


def var_size(v):
  if v in parser.tab_id:
    return parser.tab_id[v][1] * parser.tab_id[v][2]
  else:
    return 0


def var_type(v):
  if v in parser.tab_id:
    return parser.tab_id[v][3]
  else:
    return None


def p_program(p):
  """
  program : MAIN LCURLY body RCURLY
  """
  fp.write(p[3])
  print(p[3])


def p_body(p):
  """ 
  body : declarations instructions
  """
  p[0] = p[1] + 'START\n' + p[2] + 'STOP'


def p_declarations_empty(p):
  """
  declarations : 
  """
  p[0] = ""


def p_declarations(p):
  """
  declarations : declaration  declarations
  """
  p[0] = p[1] + p[2]


def p_declaration_single(p):
  """
  declaration : type VAR SEMICOLON 
  """
  status = var_new((p[2],1,1,p[1]))
  if status == -1:
    p[0] = f'ERR \"multipla declaração da variavel {p[2]}\"\nSTOP'
  else:
    p[0] = 'PUSHI 0\n'
 

def p_declaration_array(p):
  """
  declaration : type LBRACKET NUM RBRACKET VAR SEMICOLON 
  """
  status = var_new((p[5],p[3],1,p[1]))
  if status == -1:
    p[0] = f'ERR \"multipla declaração da variavel {p[2]}\"\nSTOP'
  else:
    p[0] = f'PUSHN {var_size(p[5])}\n'


def p_declaration_biarray(p):
  """
  declaration : type LBRACKET NUM RBRACKET LBRACKET NUM RBRACKET VAR SEMICOLON 

  """
  status = var_new((p[8],p[6],p[3],p[1]))
  if status == -1:
    p[0] = f'ERR \"multipla declaração da variavel {p[2]}\"\nSTOP\n'
  else:
    p[0] = f'PUSHN {var_size(p[8])}\n'


def p_type_int(p):
  """
  type : INT
  """
  p[0] = f'{p[1]}'


def p_type_float(p):
  """
  type : FLOAT
  """
  p[0] = f'{p[1]}'


def p_variable_single(p):
  """
  variable : VAR
  """
  p[0] = ('PUSHGP\n' + f'PUSHI {var_address_base(p[1])}\n',var_type(p[1]),var_size(p[1]))  


def p_variable_index_expression(p):
  """
  variable : VAR LBRACKET expression RBRACKET
  """
  p[0] = ('PUSHGP\n' + f'PUSHI {var_address_base(p[1])}\n' + p[3][0] + 
          'ADD\n',var_type(p[1]),var_size(p[1])) 


def p_variable_index_expression_expression(p):
  """
  variable : VAR LBRACKET expression RBRACKET LBRACKET expression RBRACKET
  """
  p[0] = ('PUSHGP\n' + f'PUSHI {var_address_base(p[1])}\n' + f'PUSHI {var_num_colums(p[1])}\n' 
          + p[3][0] + 'MUL\n' + 'ADD\n' + p[6][0] + 'ADD\n',var_type(p[1]),var_size(p[1])) 


def p_instructions_empty(p):
  """
  instructions :
  """
  p[0] = ""


def p_instructions(p):
  """
  instructions : instruction  instructions
  """
  p[0] = p[1] + p[2]


def p_instruction(p):
  """
  instruction : atributions SEMICOLON
  """
  p[0] = p[1]


def p_atributions_empty(p):
  """
  atributions :  
  """
  p[0] = ""


def p_atributions_single(p):
  """
  atributions : atribution 
  """
  p[0] = p[1]


def p_atributions_multiple(p):
  """
  atributions : atribution COMMA atributions 
  """
  p[0] = p[1] + p[3]


def p_instruction_atribution_expression(p):
  """
  atribution : variable EQUAL expression 
  """
  if p[1][1] == None:
    p[0] = f'ERR \"segmentation fault\\n\"\nSTOP\n'
  elif p[1][1] == 'int' and p[3][1] == 'float':
    p[0] = p[1][0] + p[3][0]+ 'FTOI\n' + 'STOREN\n'
  elif p[1][1] == 'float' and p[3][1] == 'int':
    p[0] = p[1][0] + p[3][0]+ 'ITOF\n'  + 'STOREN\n'
  else:
    p[0] = p[1][0] + p[3][0] +  'STOREN\n'


def p_instruction_atribution_condition(p):
  """
  atribution : variable EQUAL condition 
  """
  if p[1][1] == None:
    p[0] = f'ERR \"segmentation fault\\n\"\nSTOP\n'
  elif p[1][1] == 'float':
    p[0] = p[1][0] + p[3] + 'ITOF\n'  + 'STOREN\n'
  else:
    p[0] = p[1][0] + p[3] + 'STOREN\n'\


def p_expression_var(p):
  """
  expression : variable
  """
  if p[1][1] == None:
    p[0] = (f'ERR \"segmentation fault\\n\"\nSTOP\n',None)
  else:
    p[0] = (p[1][0] + 'LOADN\n',p[1][1])


def p_expression_num(p):
  """
  expression : NUM
  """
  p[0] = (f'PUSHI {p[1]}\n','int')


def p_expression_float(p):
  """
  expression : REAL
  """
  p[0] = (f'PUSHF {p[1]}\n','float')


def p_expression_between_parenthesis(p):
  """
  expression : LPAREN expression RPAREN
  """
  p[0] = p[2]


def p_expression_plus_expression(p):
  """
  expression : expression PLUS expression
  """
  if p[1][1] == 'int' and p[3][1] == 'int':
    p[0] = (p[1][0] + p[3][0] + 'ADD\n','int')
  elif p[1][1] == 'float' and p[3][1] == 'float':
    p[0] = (p[1][0] + p[3][0] + 'FADD\n','float')
  elif p[1][1] == 'int' and p[3][1] == 'float':
    p[0] = (p[1][0] + 'ITOF\n' + p[3][0] + 'FADD\n','float')
  else:
    p[0] = (p[1][0] + p[3][0] + 'ITOF\n' +'FADD\n','float')


def p_expression_minus_expression(p):
  """
  expression : expression MINUS expression
  """
  if p[1][1] == 'int' and p[3][1] == 'int':
    p[0] = (p[1][0] + p[3][0] + 'SUB\n','int')
  elif p[1][1] == 'float' and p[3][1] == 'float':
    p[0] = (p[1][0] + p[3][0] + 'FSUB\n','float')
  elif p[1][1] == 'int' and p[3][1] == 'float':
    p[0] = (p[1][0] + 'ITOF\n' + p[3][0] + 'FSUB\n','float')
  else:
    p[0] = (p[1][0] + p[3][0] + 'ITOF\n' +'FSUB\n','float')


def p_expression_mul_expression(p):
  """
  expression : expression MUL expression
  """
  if p[1][1] == 'int' and p[3][1] == 'int':
    p[0] = (p[1][0] + p[3][0] + 'MUL\n','int')
  elif p[1][1] == 'float' and p[3][1] == 'float':
    p[0] = (p[1][0] + p[3][0] + 'FMUL\n','float')
  elif p[1][1] == 'int' and p[3][1] == 'float':
    p[0] = (p[1][0] + 'ITOF\n' + p[3][0] + 'FMUL\n','float')
  else:
    p[0] = (p[1][0] + p[3][0] + 'ITOF\n' +'FMUL\n','float')


def p_expression_div_expression(p):
  """
  expression : expression DIV expression
  """
  if p[1][1] == 'int' and p[3][1] == 'int':
    p[0] = (p[1][0] + p[3][0] + 'DIV\n','int')
  elif p[1][1] == 'float' and p[3][1] == 'float':
    p[0] = (p[1][0] + p[3][0] + 'FDIV\n','float')
  elif p[1][1] == 'int' and p[3][1] == 'float':
    p[0] = (p[1][0] + 'ITOF\n' + p[3][0] + 'FDIV\n','float')
  else:
    p[0] = (p[1][0] + p[3][0] + 'ITOF\n' +'FDIV\n','float')


def p_expression_mod_expression(p):
  """
  expression : expression MOD expression
  """
  if p[1][1] == 'int' and p[3][1] == 'int':
    p[0] = (p[1][0] + p[3][0] + 'MOD\n','int')
  elif p[1][1] == 'float' and p[3][1] == 'float':
    p[0] = (p[1][0] + 'FTOI\n' + p[3][0] + 'FTOI\n' + 'MOD\n','int')
  elif p[1][1] == 'int' and p[3][1] == 'float':
    p[0] = (p[1][0] + p[3][0] + 'FTOI\n' + 'MOD\n','int')
  else:
    p[0] = (p[1][0] + 'FTOI\n' + p[3][0] +'MOD\n','int')


def p_instruction_while(p):
  """
  instruction : WHILE LPAREN condition RPAREN LCURLY instructions RCURLY
  """
  p[0] = f'B{parser.labels}:\n' + p[3] + 'JZ ' + f'E{parser.labels}\n' + p[6] + 
  f'JUMP B{parser.labels}\n' + f'E{parser.labels}:\n' 
  parser.labels +=1


def p_instruction_for(p):
  """
  instruction : FOR LPAREN atributions SEMICOLON condition SEMICOLON atributions RPAREN LCURLY instructions RCURLY
  """
  p[0] = p[3] + f'B{parser.labels}:\n' + p[5] + 'JZ ' + f'E{parser.labels}\n' + p[10] + p[7] + 
  f'JUMP B{parser.labels}\n' + f'E{parser.labels}:\n' 
  parser.labels +=1


def p_instruction_if(p):
  """
  instruction : IF LPAREN condition RPAREN LCURLY instructions RCURLY
  """
  p[0] = p[3] + 'JZ '+ f'E{parser.labels}\n' + p[6] + f'E{parser.labels}:\n' 
  parser.labels +=1


def p_instruction_if_else(p):
  """
  instruction : IF LPAREN condition RPAREN LCURLY instructions RCURLY ELSE LCURLY instructions RCURLY
  """
  p[0] = p[3] + 'JZ '+ f'E{parser.labels}\n' + p[6] + f'JUMP F{parser.labels}\n' +
  f'E{parser.labels}:\n' + p[10] + f'F{parser.labels}:\n' 
  parser.labels +=1

  
def p_condition_expression_eqeq_expression(p):
  """
  condition : expression EQEQ expression
  """
  if p[1][1] == 'int' and p[3][1] == 'float':
    p[0] = p[1][0] + 'ITOF\n' + p[3][0] + 'EQUAL\n'
  elif p[1][1] == 'float' and p[3][1] == 'int':
    p[0] = p[1][0] + p[3][0] + 'ITOF\n' + 'EQUAL\n'
  else:
    p[0] = p[1][0] + p[3][0] + 'EQUAL\n'
  

def p_condition_expression_diff_expression(p):
  """
  condition : expression DIFF expression
  """
  if p[1][1] == 'int' and p[3][1] == 'float':
    p[0] = p[1][0] + 'ITOF\n' + p[3][0] + 'EQUAL\nNOT\n'
  elif p[1][1] == 'float' and p[3][1] == 'int':
    p[0] = p[1][0] + p[3][0] + 'ITOF\n' + 'EQUAL\nNOT\n'
  else:
    p[0] = p[1][0] + p[3][0] + 'EQUAL\nNOT\n'


def p_condition_expression_greater_expression(p):
  """
  condition : expression GREATER expression
  """
  if p[1][1] == 'float' and p[3][1] == 'float':
    p[0] = p[1][0] + p[3][0] + 'FSUP\n' + 'FTOI\nPUSHI 0\nSUP\n'
  elif p[1][1] == 'int' and p[3][1] == 'float':
    p[0] = p[1][0] + 'ITOF\n' + p[3][0] + 'FSUP\n' + 'FTOI\nPUSHI 0\nSUP\n'
  elif p[1][1] == 'float' and p[3][1] == 'int':
    p[0] = p[1][0] + p[3][0] + 'ITOF\n' + 'FSUP\n' + 'FTOI\nPUSHI 0\nSUP\n'
  else:
    p[0] = p[1][0] + p[3][0] + 'SUP\n'


def p_condition_expression_lesser_expression(p):
  """
  condition : expression LESSER expression
  """
  if p[1][1] == 'float' and p[3][1] == 'float':
    p[0] = p[1][0] + p[3][0] + 'FINF\n' + 'FTOI\nPUSHI 0\nSUP\n'

  elif p[1][1] == 'int' and p[3][1] == 'float':
    p[0] = p[1][0] + 'ITOF\n' + p[3][0] + 'FINF\n' + 'FTOI\nPUSHI 0\nSUP\n'
  
  elif p[1][1] == 'float' and p[3][1] == 'int':
    p[0] = p[1][0] + p[3][0] + 'ITOF\n' + 'FINF\n' + 'FTOI\nPUSHI 0\nSUP\n'

  else:
    p[0] = p[1][0] + p[3][0] + 'INF\n'


def p_condition_expression_greateq_expression(p):
  """
  condition : expression GREATEQ expression
  """
  if p[1][1] == 'float' and p[3][1] == 'float':
    p[0] = p[1][0] + p[3][0] + 'FSUPEQ\n' + 'FTOI\nPUSHI 0\nSUP\n'

  elif p[1][1] == 'int' and p[3][1] == 'float':
    p[0] = p[1][0] + 'ITOF\n' + p[3][0] + 'FSUPEQ\n' + 'FTOI\nPUSHI 0\nSUP\n'
  
  elif p[1][1] == 'float' and p[3][1] == 'int':
    p[0] = p[1][0] + p[3][0] + 'ITOF\n' + 'FSUPEQ\n' + 'FTOI\nPUSHI 0\nSUP\n'

  else:
    p[0] = p[1][0] + p[3][0] + 'SUPEQ\n'


def p_condition_expression_lesseq_expression(p):
  """
  condition : expression LESSEQ expression
  """
  if p[1][1] == 'float' and p[3][1] == 'float':
    p[0] = p[1][0] + p[3][0] + 'FINFEQ\n' + 'FTOI\nPUSHI 0\nSUP\n'
  elif p[1][1] == 'int' and p[3][1] == 'float':
    p[0] = p[1][0] + 'ITOF\n' + p[3][0] + 'FINFEQ\n' + 'FTOI\nPUSHI 0\nSUP\n'
  elif p[1][1] == 'float' and p[3][1] == 'int':
    p[0] = p[1][0] + p[3][0] + 'ITOF\n' + 'FINFEQ\n' + 'FTOI\nPUSHI 0\nSUP\n'
  else:
    p[0] = p[1][0] + p[3][0] + 'INFEQ\n'


def p_condition_num(p):
  """
  condition : NUM
  """
  p[0] = f'PUSHI {p[1]}\n'


def p_condition_real(p):
  """
  condition : REAL
  """
  p[0] = f'PUSHF {p[1]}\nPUSHF 0.0\nFSUP\nFTOI\nPUSHI 0\nSUP\n'


def p_condition_var(p):
  """
  condition : variable
  """
  if p[1][1] == None:
    p[0] = f'ERR \"segmentation fault\\n\"\nSTOP\n'
  elif p[1][1] == 'float':
    
    p[0] = p[1][0] + 'LOADN\nPUSHF 0.0\nFSUP\nFTOI\nPUSHI 0\nSUP\n'
  else:
    p[0] = p[1][0] + 'LOADN\nPUSHI 0\nSUP\n'


def p_instruction_scan(p):
  """
  instruction : SCAN LPAREN variable RPAREN SEMICOLON
  """
  if p[3][1] == None:
    p[0] = f'ERR \"segmentation fault\\n\"\nSTOP\n'
  elif p[3][1] == 'int':
    p[0] = p[3][0]  + 'READ\nATOI\nSTOREN\n'
  else:
    p[0] = p[3][0]  + 'READ\nATOF\nSTOREN\n'
  

def p_instruction_print_var(p):
  """
  instruction : PRINT LPAREN variable RPAREN SEMICOLON
  """
  if p[3][1] == None:
    p[0] = f'ERR \"segmentation fault\\n\"\nSTOP\n'
  elif p[3][1] == 'int':
    p[0] = p[3][0]  + 'LOADN\nWRITEI\n'
  else:
    p[0] = p[3][0]  + 'LOADN\nWRITEF\n'


def p_instruction_println_var(p):
  """
  instruction : PRINTLN LPAREN variable RPAREN SEMICOLON
  """
  if p[3][1] == None:
    p[0] = f'ERR \"segmentation fault\\n\"\nSTOP\n'
  elif p[3][1] == 'int':
    p[0] = p[3][0]  + 'LOADN\nWRITEI\n'
    p[0] += 'PUSHS\"\\n\"\nWRITES\n'
  else:
    p[0] = p[3][0]  + 'LOADN\nWRITEF\n'
    p[0] += 'PUSHS\"\\n\"\nWRITES\n'

  
def p_instruction_print_string(p):
  """
  instruction : PRINT LPAREN STRING RPAREN SEMICOLON
  """
  p[0] = f'PUSHS {p[3]}\nWRITES\n'


def p_instruction_println_string(p):
  """
  instruction : PRINTLN LPAREN STRING RPAREN SEMICOLON
  """
  p[0] = f'PUSHS {p[3]}\nWRITES\nPUSHS\"\\n\"\nWRITES\n'


def p_error(p):
  print("Syntax error!")
  parser.success = False


parser = yacc.yacc()
parser.success = True

parser.prox_address = 0
parser.tab_id = {}
parser.labels = 0


if len(sys.argv)!=2 and len(sys.argv)!=3:
    print('Invalid number of arguments!')
    sys.exit(0)
else:
    file_input = sys.argv[1]

if not os.path.exists(file_input):
  print(f"File \"{file_input}\" not found!")
  sys.exit(0)


fp = open(file_input, 'r')
source = fp.read()
fp.close()


if len(sys.argv)==3:
    file_output = sys.argv[2]
else:
  file_output = "a.vm"


fp = open(file_output,"w")

parser.parse(source)


if parser.success:
  print("Parsing successfully completed!")

fp.close()
\end{verbatim}
\end{scriptsize}



\lstinputlisting[caption={Exemplo de uma importação}, label={lstExe2}]{listagemImportadaLayout.l} %input de um ficheiro da listagem

%-- Fim do documento -- inserção das referencias bibliográficas

%\bibliographystyle{plain} % [1] Numérico pela ordem de citação ou ordem alfabetica
\bibliographystyle{alpha} % [Hen18] abreviação do apelido e data da publicação
%\bibliographystyle{apalike} % (Araujo, 2018) apelido e data da publicação
                             % --para usar este estilo descomente no inicio o comando \usepackage{apalike}

\bibliography{bibLayout} %input do ficheiro de referencias bibliograficas

\end{document} 