
\documentclass[11pt,a4paper]{report}%especifica o tipo de documento que tenciona escrever: carta, artigo, relatório... neste caso é um relatório
% [11pt,a4paper] Define o tamanho principal das letras do documento. caso não especifique uma delas, é assumido 10pt
% a4paper -- Define o tamanho do papel.

\usepackage[portuges]{babel}%Babel -- irá activar automaticamente as regras apropriadas de hifenização para a língua todo o
                                   %-- o texto gerado é automaticamente traduzido para Português.
                                   %  Por exemplo, “chapter” irá passar a “capítulo”, “table of contents” a “conteúdo”.
                                   % portuges -- específica para o Português.
\usepackage[utf8]{inputenc} % define o encoding usado texto fonte (input)--usual "utf8" ou "latin1

\usepackage{graphicx} %permite incluir graficos, tabelas, figuras
\usepackage{url} % para utilizar o comando \url{}
\usepackage{enumerate} %permite escolher, nas listas enumeradas, se os iems sao marcados com letras ou numeros-romanos em vez de numeracao normal

%\usepackage{apalike} % gerar biliografia no estilo 'named' (apalike)

\usepackage{color} % Para escrever em cores

\usepackage{multirow} %tabelas com multilinhas
\usepackage{array} %formatação especial de tabelas em array

\usepackage[pdftex]{hyperref} % transformar as referências internas do seu documento em hiper-ligações.

%Exemplos de fontes -- nao e vulgar mudar o tipo de fonte
%\usepackage{tgbonum} % Fonte de letra: TEX Gyre Bonum
%\usepackage{lmodern} % Fonte de letra: Latin Modern Sans Serif
%\usepackage{helvet}  % Fonte de letra: Helvetica
%\usepackage{charter} % Fonte de letra:Charter

\definecolor{saddlebrown}{rgb}{0.55, 0.27, 0.07} % para definir uma nova cor, neste caso 'saddlebrown'

\usepackage{listings}  % para utilizar blocos de texto verbatim no estilo 'listings'
%paramerização mais vulgar dos blocos LISTING - GENERAL
\lstset{
	basicstyle=\small, %o tamanho das fontes que são usadas para o código
	numbers=left, % onde colocar a numeração da linha
	numberstyle=\tiny, %o tamanho das fontes que são usadas para a numeração da linha
	numbersep=5pt, %distancia entre a numeração da linha e o codigo
	breaklines=true, %define quebra automática de linha
    frame=tB,  % caixa a volta do codigo
	mathescape=true, %habilita o modo matemático
	escapeinside={(*@}{@*)} % se escrever isto  aceita tudo o que esta dentro das marcas e nao altera
}
%
%\lstset{ %
%	language=Java,							% choose the language of the code
%	basicstyle=\ttfamily\footnotesize,		% the size of the fonts that are used for the code
%	keywordstyle=\bfseries,					% set the keyword style
%	%numbers=left,							% where to put the line-numbers
%	numberstyle=\scriptsize,				% the size of the fonts that are used for the line-numbers
%	stepnumber=2,							% the step between two line-numbers. If it's 1 each line
%											% will be numbered
%	numbersep=5pt,							% how far the line-numbers are from the code
%	backgroundcolor=\color{white},			% choose the background color. You must add \usepackage{color}
%	showspaces=false,						% show spaces adding particular underscores
%	showstringspaces=false,					% underline spaces within strings
%	showtabs=false,							% show tabs within strings adding particular underscores
%	frame=none,								% adds a frame around the code
%	%abovecaptionskip=-.8em,
%	%belowcaptionskip=.7em,
%	tabsize=2,								% sets default tabsize to 2 spaces
%	captionpos=b,							% sets the caption-position to bottom
%	breaklines=true,						% sets automatic line breaking
%	breakatwhitespace=false,				% sets if automatic breaks should only happen at whitespace
%	title=\lstname,							% show the filename of files included with \lstinputlisting;
%											% also try caption instead of title
%	escapeinside={\%*}{*)},					% if you want to add a comment within your code
%	morekeywords={*,...}					% if you want to add more keywords to the set
%}

\usepackage{xspace} % deteta se a seguir a palavra tem uma palavra ou um sinal de pontuaçao se tiver uma palavra da espaço, se for um sinal de pontuaçao nao da espaço

\parindent=0pt %espaço a deixar para fazer a  indentação da primeira linha após um parágrafo
\parskip=2pt % espaço entre o parágrafo e o texto anterior

\setlength{\oddsidemargin}{-1cm} %espaço entre o texto e a margem
\setlength{\textwidth}{18cm} %Comprimento do texto na pagina
\setlength{\headsep}{-1cm} %espaço entre o texto e o cabeçalho
\setlength{\textheight}{23cm} %altura do texto na pagina

% comando '\def' usado para definir abreviatura (macros)
% o primeiro argumento é o nome do novo comando e o segundo entre chavetas é o texto original, ou sequência de controle, para que expande
\def\darius{\textsf{Darius}\xspace}
\def\antlr{\texttt{AnTLR}\xspace}
\def\pe{\emph{Publicação Eletrónica}\xspace}
\def\titulo#1{\section{#1}}    %no corpo do documento usa-se na forma '\titulo{MEU TITULO}'
\def\super#1{{\em Supervisor: #1}\\ }
\def\area#1{{\em \'{A}rea: #1}\\[0.2cm]}
\def\resumo{\underline{Resumo}:\\ }

%\input{LPgeneralDefintions} %permite ler de um ficheiro de texto externo mais definições

\title{Processamento de Linguaguens e Compiladores (3º ano LCC)\\
       \textbf{Trabalho Prático 1}\\ Relatório de Desenvolvimento \\ \textbf{Grupo 17}\\ Problema 3
       } %Titulo do documento
%\title{Um Exemplo de Artigo em \LaTeX}
\author{José Pedro Gomes Ferreira\\ A91636 \\  \and Pedro Paulo Costa Pereira\\ A88062
         \and Tiago André Oliveira Leite\\ A91693
       } %autores do documento
\date{\today} %data

\begin{document} % corpo do documento
\maketitle % apresentar titulo, autor e data


\tableofcontents % Insere a tabela de indice
%\listoffigures % Insere a tabela de indice figuras
%\listoftables % Insere a tabela de indice tabelas


%%%%%%%%%%%%%%%%%%%%%%
%%%%%Capitulo1%%%%%%%%
%%%%%%%%%%%%%%%%%%%%%%


\chapter{Introdução} \label{chap:intro} %referência cruzada

Este documento tem como objetivo explicar a solução que implementamos para a resolução do probelma proposto no âmbito da unidade curricular de Processsamento de Linguagens e Compiladores. \\ \\
O problema consiste em implementar uma linguagem de programação imperativa simples, com regras sintaticas definidas pelo grupo. \\ \\ Para o desenvolvimento da nosso linguagem tivemos que definir uma gramática independente de contexto \textbf{GIC} e desenvolver um compilador que gera \textbf{pseudo-código}, \verb Assembly \ para uma Máquina Virtual, VM, que nos foi fornecida. \\ \\
No desenvolvimento do programa, para alem do conhecimento adquirido nas aulas, utilizamos os módulos 'Yacc/ Lex' do 'PLY/Python'. \\ \\ 
Esperamos que o resultado final cumpra todos os requisitos.  


%%%%%%%%%%%%%%%%%%%%%%
%%%%%Capitulo2%%%%%%%%
%%%%%%%%%%%%%%%%%%%%%%



\chapter{Problema Proposto} \label{chap:ProbelemaProposto} %capitulo e referencia cruzada

\section{Descrição} \label{sec:Descricao} %seccao e referencia cruzada
Pretende-se que comece por definir uma linguagem de programação imperativa simples, a seu gosto.
Apenas deve ter em consideração que essa linguagem terá de permitir:
\begin{itemize}

\item \textit{declarar} variáveis atómicas do tipo \textit{inteiro}, com os quais se podem realizar as habituais operações aritméticas, relacionais e lógicas;

\item \textit{efetuar} instruções algorı́tmicas básicas como a \textit{atribuição do valor de expressões numéricas a variáveis};

\item \textit{ler} do \textit{standard input} e \textit{escrever} no \textit{standard output};

\item \textit{efetuar} instruções \textit{condicionais} para controlo do fluxo de execução;

\item \textit{efetuar} instruções cı́clicas para controlo do fluxo de execução, permitindo o seu aninhamento.\\
\underline{Note} que deve implementar pelo menos o ciclo \textbf{while-do}, \textbf{repeat-until} ou \textbf{for-do}.
\end{itemize}

Adicionalmente deve ainda suportar, à sua escolha, uma das duas funcionalidades seguintes:
\begin{itemize}
\item \textit{declarar e manusear} variáveis estruturadas do tipo \textit{array (a 1 ou 2 dimensões)} de inteiros, em relação aos quais é apenas permitida a operação de indexação (ı́ndice inteiro);
\item \textit{definir e invocar subprogramas} sem parâmetros mas que possam retornar um resultado do tipo inteiro.
\end{itemize}


\section{Requisitos} \label{sec:Requesitos}
\begin{itemize}
  \item Utilização da linguagem Python.
  \item Resolver o problema recurso aos módulos 'Yacc/ Lex' do 'PLY/Python'.
  \item Gerar \textbf{pseudo-código}, \verb Assembly \ da Máquina Virtual VM fornecida.
  \item Preparar um conjunto de testes de modo a ver
o código Assembly gerado bem como o programa a correr na máquina virtual VM. Este conjunto terá de conter, no mı́nimo, os 4 primeiros exemplos abaixo e um dos 2 últimos conforme a escolha de funcionalidades da linguagem:
\begin{itemize}
\item ler 4 números e dizer se podem ser os lados de um quadrado;
\item ler um inteiro N, depois ler N números e escrever o menor deles;
\item ler N (constante do programa) números e calcular e imprimir o seu produtório;
\item contar e imprimir os números impares de uma sequência de números naturais;
\item ler e armazenar N números num array; imprimir os valores por ordem inversa;
\item invocar e usar num programa seu uma função ’potencia()’, que começa por ler do input a base $B$ e o expoente $E$ e retorna o valor  $B^E$.
\end{itemize}
\end{itemize}

%%%%%%%%%%%%%%%%%%%%%%
%%%%%Capitulo3%%%%%%%%
%%%%%%%%%%%%%%%%%%%%%%

\chapter{Concepção/desenho da Resolução}
\section{Organização e estrutura}
O nosso trabalho pode ser divido em \textbf{4} partes:
\begin{itemize}
\item Definição da \textbf{GIC} que define a estrura sintática da nosso liguaguem.
\item Construção do analizador léxico \verb lexer .
\item Construção do analizador sintático \verb parser .
\item Conversão das instruções para código \verb Assembly \ da  VM.    
\end{itemize}
Todas as funcionalidades descritas neste capítulo podem ser encontradas no anexo A do documento.\\
No nosso trabalho optamos por implementar a funcionalidade de textit{declarar e manusear} variáveis estruturadas do tipo \textit{array (a 1 ou 2 dimensões)} de inteiros, em relação aos quais é apenas permitida a operação de indexação (ı́ndice inteiro).\\

\section{GIC}
A nossa linguagem é gerada pela seguinte grámatica independente de contexto:

\begin{verbatim}
program      : MAIN LCURLY body RCURLY

body         : declarations instructions

declarations : 
             | declaration  declarations

declaration  : type VAR SEMICOLON
             | type LBRACKET NUM RBRACKET VAR SEMICOLON
             | type LBRACKET NUM RBRACKET LBRACKET NUM RBRACKET VAR SEMICOLON

type         : INT
             | FLOAT

instructions :
             | instruction  instructions

instruction  : atributions SEMICOLON
             | WHILE LPAREN condition RPAREN LCURLY instructions RCURLY
             | FOR LPAREN atributions SEMICOLON condition SEMICOLON atributions RPAREN LCURLY 
               instructions RCURLY
             | IF LPAREN condition RPAREN LCURLY instructions RCURLY
             | IF LPAREN condition RPAREN LCURLY instructions RCURLY ELSE LCURLY 
               instructions RCURLY
             | SCAN LPAREN variable RPAREN SEMICOLON
             | PRINT LPAREN variable RPAREN SEMICOLON
             | PRINTLN LPAREN variable RPAREN SEMICOLON
             | PRINT LPAREN STRING RPAREN SEMICOLON
             | PRINTLN LPAREN STRING RPAREN SEMICOLON

atributions  : 
             | atribution
             | atribution COMMA atributions

atribution   : variable EQUAL expression 
             | variable EQUAL condition

variable     : VAR
             | VAR LBRACKET expression RBRACKET
             | VAR LBRACKET expression RBRACKET LBRACKET expression RBRACKET

expression   : variable
             | NUM
             | REAL
             | LPAREN expression RPAREN
             | expression PLUS expression
             | expression MINUS expression
             | expression MUL expression
             | expression DIV expression
             | expression MOD expression

condition    : expression EQEQ expression
             | expression DIFF expression
             | expression GREATER expression
             | expression LESSER expression
             | expression GREATEQ expression
             | expression LESSEQ expression
             | NUM
             | REAL
             | variable
\end{verbatim}


\section{Lexer}
O analisador léxico, \verb lexer, vai ser o responsavel por 'capturar' os simbolos terminais, \verb tokens, da nossa linguagem através de expressões regulares. Para a implementação do analisador léxico utilizamos o módulo 'Lex' do 'PLY/Python'.\\
De seguida apresentamos os \verb tokens \ da nossa linguagem e respetivas expressão regular.
\begin{verbatim}

SEMICOLON : ';'
COMMA     : ','
LCURLY    : '\{'
RCURLY    : '\}'
LPAREN    : '\('
RPAREN    : '\)'
LBRACKET  : '\['
RBRACKET  : '\]'
FLOAT     : 'float'
INT       : 'int'
MAIN      : 'main'
WHILE     : 'while'
FOR       : 'for'
IF        : 'if'
ELSE      : 'else'
PRINTLN   : 'println'
PRINT     : 'print'
SCAN      : 'scan'
STRING    : '"([^"]|(\\n))*"'
REAL      : '-?([1-9][0-9]*\.[0-9]+|0\.[0-9]+)'
NUM       : '-?\d+'
EQUAL     : '\=\='
DIFF      : '\!\='
GREATEQ   : '\>\='
LESSEQ    : '\<\='
GREATER   : '\>'
LESSER    : '\<'
EQEQ      : '\='
PLUS      : '\+'
MINUS     : '\-'
MUL       : '\*'
DIV       : '\/'
MOD       : '\%'
VAR       : '\w+'
\end{verbatim}   

\section{Parser e geração do codigo Assembly da VM}
O analisador sintático, \verb parser, vai ser o responsavel por verificar se o código que foi escrito na nossa linguagem está sintaticamente correto, isto é, respeitas as regras gramaticais definidas. Caso não haja erros sintáticos o \parser converte o código da nossa linguagem em codigo \verb Assembly da máquina virtual.


%%%%%%%%%%%%%%%%%%%%%%
%%%%%Capitulo4%%%%%%%%
%%%%%%%%%%%%%%%%%%%%%%

\chapter{Demonstração de Funcionamento}
\section{Ficheiro: 'clav-users'}
\subsection{Conteúdo do ficheiro.}

\begin{verbatim}
Élia Cristina Viegas Pedro :: epedro@ccdr-alg.pt :: ent_CCDR-Alg :: 1 :: 0
Formação DGLAB :: lurdes.almeida@dglab.gov.pt :: ent_DGLAB :: 3.5 :: 0
Nuno Filipe Casas Novas :: nuno.novas@ccdr-lvt.pt :: ent_CCDR-LVT :: 1 :: 0
Sónia Patrícia Pinheiro Reis :: sonia.reis@icnf.pt :: ent_ICNF :: 1 :: 0
Sónia Isabel Ferreira Gonçalves Negrão :: sonia.negrao@cm-albufeira.pt :: ent_CMABF :: 1 :: 0
Filipe Ferreira Cardoso Leitão :: arquivo@cm-spsul.pt :: ent_CMSPS :: 1 :: 0
Ana Lúcia Cabrita Guerreiro :: alucia@ccdr-alg.pt :: ent_CCDR-Alg :: 1 :: 0
Alda do Carmo Namora Soares de Andrade :: aandrade@letras.ulisboa.pt :: ent_FLUL :: 1 :: 0
Ricardo Almeida :: ricardo.almeida@dgeg.gov.pt :: ent_DGEG :: 1 :: 0
Sandra Cristina Patrício da Silva :: spatricio@mun-sines.pt :: ent_CMSNS :: 1 :: 0
Cátia João Matias Trindade :: catia.trindade@dglab.gov.pt :: ent_DGLAB :: 4 :: 0
Ricardo Canela :: tyty@tyty.pt :: ent_BdP :: 3 :: 0
Cátia Trindade :: matiasjcatia@gmail.com :: ent_DGLAB :: 4 :: 0
Miguel Ferreira :: mferreira@keep.pt :: ent_KEEP :: 7 :: 0
Fernando Manuel Antunes Marques da Silva :: fernando.marques.silva@marinha.pt :: ent_STI-M :: 1 :: 0
Ana Maria Teixeira Gaspar :: ana.gaspar@sgmf.gov.pt :: ent_SGMF :: 1 :: 0
Maria Matos de Almeida Talhada Correia :: MariaMatos.Correia@icnf.pt :: ent_ICNF :: 1 :: 0
Cármen Isabel Amador Francisco :: carmem@mun-sines.pt :: ent_CMSNS :: 1 :: 0
Maria Leonor da Conceição Fresco Franco :: leonor.mina@ccdr-lvt.pt :: ent_CCDR-LVT :: 1 :: 0
Maria Rita Gago :: m-rita.gago@dglab.gov.pt :: ent_DGLAB :: 6 :: 3
José Carlos Leite Ramalho :: jcr@di.uminho.pt :: ent_A3ES :: 7 :: 3
Regina Neves Lopes :: Regina.Lopes@sgmf.gov.pt :: ent_SGMF :: 1 :: 0
João Paulo de Melo Esteves Pereira :: joao.pereira@apambiente.pt :: ent_APA :: 1 :: 0
António José Morim Brandão :: Antonio.brandao@metrodoporto.pt :: ent_MdP :: 1 :: 0
Alexandra Lourenço :: alexandra.lourenco@dglab.gov.pt :: ent_DGLAB :: 7 :: 3
Rui Araújo :: pg38425@alunos.uminho.pt :: ent_II :: 7 :: 9
Filipa Carvalho :: filipa.carvalho@dglab.gov.pt :: ent_DGLAB :: 4 :: 0
clara cristina rainho viegas :: clara.viegas@dglab.gov.pt :: ent_DGLAB :: 4 :: 0
Madalena Ribeiro :: madalena.ribeiro@dglab.gov.pt :: ent_DGLAB :: 6 :: 0
Zélia Gomes :: zelia.gomes@dglab.gov.pt :: ent_DGLAB :: 4 :: 0
Frederico Pinto :: frederico21pinto@hotmail.com :: ent_ACSS :: 7 :: 0
Pedro Penteado :: pedro.penteado@dglab.gov.pt :: ent_DGLAB :: 5 :: 1
Madalena Ribeiro :: madalena.ribeiro07@gmail.com :: ent_DGLAB :: 4 :: 2
Alexandre Teixeira :: a73547@alunos.uminho.pt :: ent_A3ES :: 7 :: 0
octavio :: octaviojmaia@gmail.com :: ent_A3ES :: 2 :: 0
José Carlos Martins :: a78821@alunos.uminho.pt :: ent_A3ES :: 7 :: 0
Silvestre Lacerda :: silvestre.lacerda@dglab.gov.pt :: ent_DGLAB :: 5 :: 0
Design-DGLAB :: a75536@alunos.uminho.pt :: ent_DGLAB :: 7 :: 0
João Pimentel :: A80874@alunos.uminho.pt :: ent_A3ES :: 7 :: 0
Maria Celeste Pereira :: m-celeste.pereira@dglab.gov.pt :: ent_DGLAB :: 4 :: 0
CLAV-migrator :: a74036@alunos.uminho.pt :: ent_A3ES :: 7 :: 0
Alexadre Teixeira :: alex@keep.pt :: ent_KEEP :: 7 :: 0
Carlos Barbosa :: A82324@alunos.uminho.pt :: ent_A3ES :: 7 :: 0
Rui Araújo Simples :: 0.rffa.0@gmail.com :: ent_LNEC :: 1 :: 0
Manuel Monteiro :: a74036@alunos.uminhopt :: ent_A3ES :: 7 :: 0
Rui Araújo Entidade :: ruifilipearaujo@hotmail.com :: ent_AAN :: 1 :: 0
Duarte Freitas :: a63129@alunos.uminho.pt :: ent_A3ES :: 7 :: 0
Maria José Maciel Chaves :: m-jose.chaves@dglab.gov.pt :: ent_DGLAB :: 4 :: 0
jcm :: jcm@live.com :: ent_AAN :: 1 :: 0
José Carlos Leite Ramalho :: jcr@dglab.pt :: ent_DGLAB :: 7 :: 0
Alexandra Testes :: m-alexandra.lourenco@dglab.gov.pt :: ent_A3ES :: 1 :: 0
Paulo Lima :: plima@keep.pt :: ent_KEEP :: 7 :: 0
Alexandra Maria Alves Coutinho Rodrigues :: acoutinh@utad.pt :: ent_UTAD :: 1 :: 0
Aluno de DAW2020 :: leo.ramalho99@gmail.com :: ent_A3ES :: 2 :: 0
PRI2020-teste :: pri2020@teste.uminho.pt :: ent_A3ES :: 2 :: 0
DAW2020-teste :: daw2020@teste.uminho.pt :: ent_A3ES :: 2 :: 0
jcr-rep-entidade :: zzglider@gmail.com :: ent_A3ES :: 1 :: 0
Carlos Matoso :: cmatoso@ambisig.com :: ent_IEFP :: 2 :: 0
Joana Braga :: jbraga@ambisig.com :: ent_IEFP :: 2 :: 0
David Ferreira :: david.ferreira@ambisig.com :: ent_IEFP :: 2 :: 0
\end{verbatim}


\subsection{Execução e menu.}
\begin{scriptsize}
\begin{verbatim}
>> python3 main.py clav-users.txt

*** Selecione Opção ***

1. Listagem com o nome e a entidade do utilizador, ordenada alfabeticamente por nome.
2. Lista ordenada alfabeticamente das entidades referenciadas, indicando, para cada uma, quantos utilizadores estão registados.
3. Distribuição de utilizadores por níveis de acesso.
4. Utilizadores, agrupados por entidade, ordenada primeiro pela entidade e dentro desta pelo nome.
5. Mostrar alguns indicadores.
6. Imprimir os 20 primeiros registos num novo ficheiro de output mas em formato jason.
0. Sair.
>> 
\end{verbatim}
\end{scriptsize}

\subsection{Nome dos utilizadores ordenados e respetiva entidade.}

\begin{verbatim}
>> 1

*** Utilizador : Entidade(s) ***

Alda do Carmo Namora Soares de Andrade : ent_FLUL
Alexadre Teixeira : ent_KEEP
Alexandra Lourenço : ent_DGLAB
Alexandra Maria Alves Coutinho Rodrigues : ent_UTAD
Alexandra Testes : ent_A3ES
Alexandre Teixeira : ent_A3ES
Aluno de DAW2020 : ent_A3ES
Ana Lúcia Cabrita Guerreiro : ent_CCDR
Ana Maria Teixeira Gaspar : ent_SGMF
António José Morim Brandão : ent_MdP
Carlos Barbosa : ent_A3ES
Carlos Matoso : ent_IEFP
Cármen Isabel Amador Francisco : ent_CMSNS
Cátia João Matias Trindade : ent_DGLAB
Cátia Trindade : ent_DGLAB
clara cristina rainho viegas : ent_DGLAB
CLAV-migrator : ent_A3ES
David Ferreira : ent_IEFP
DAW2020-teste : ent_A3ES
Design-DGLAB : ent_DGLAB
Duarte Freitas : ent_A3ES
Élia Cristina Viegas Pedro : ent_CCDR
Fernando Manuel Antunes Marques da Silva : ent_STI
Filipa Carvalho : ent_DGLAB
Filipe Ferreira Cardoso Leitão : ent_CMSPS
Formação DGLAB : ent_DGLAB
Frederico Pinto : ent_ACSS
jcm : ent_AAN
jcr-rep-entidade : ent_A3ES
Joana Braga : ent_IEFP
João Paulo de Melo Esteves Pereira : ent_APA
João Pimentel : ent_A3ES
José Carlos Leite Ramalho : ent_A3ES, ent_DGLAB
José Carlos Martins : ent_A3ES
Madalena Ribeiro : ent_DGLAB, ent_DGLAB
Manuel Monteiro : ent_A3ES
Maria Celeste Pereira : ent_DGLAB
Maria José Maciel Chaves : ent_DGLAB
Maria Leonor da Conceição Fresco Franco : ent_CCDR
Maria Matos de Almeida Talhada Correia : ent_ICNF
Maria Rita Gago : ent_DGLAB
Miguel Ferreira : ent_KEEP
Nuno Filipe Casas Novas : ent_CCDR
octavio : ent_A3ES
Paulo Lima : ent_KEEP
Pedro Penteado : ent_DGLAB
PRI2020-teste : ent_A3ES
Regina Neves Lopes : ent_SGMF
Ricardo Almeida : ent_DGEG
Ricardo Canela : ent_BdP
Rui Araújo : ent_II
Rui Araújo Entidade : ent_AAN
Rui Araújo Simples : ent_LNEC
Sandra Cristina Patrício da Silva : ent_CMSNS
Silvestre Lacerda : ent_DGLAB
Sónia Isabel Ferreira Gonçalves Negrão : ent_CMABF
Sónia Patrícia Pinheiro Reis : ent_ICNF
Zélia Gomes : ent_DGLAB

Pressione Enter!
>> 
\end{verbatim}


\subsection{Entidades ordenadas e número de utilizadores registados em cada uma.}

\begin{verbatim}
>> 2

*** Entidade : Nº Utilizadores ***

ent_A3ES : 14
ent_AAN : 2
ent_ACSS : 1
ent_APA : 1
ent_BdP : 1
ent_CCDR : 4
ent_CMABF : 1
ent_CMSNS : 2
ent_CMSPS : 1
ent_DGEG : 1
ent_DGLAB : 16
ent_FLUL : 1
ent_ICNF : 2
ent_IEFP : 3
ent_II : 1
ent_KEEP : 3
ent_LNEC : 1
ent_MdP : 1
ent_SGMF : 2
ent_STI : 1
ent_UTAD : 1

Pressione Enter!
>> 
\end{verbatim}



\subsection{Distribuiçao dos utilizadores por nível de acesso.}

\begin{verbatim}
>> 3

*** Nível de Acesso : Distribuição ***

Nível 1 : 40%
* Alda do Carmo Namora Soares de Andrade
* Alexandra Maria Alves Coutinho Rodrigues
* Alexandra Testes
* Ana Lúcia Cabrita Guerreiro
* Ana Maria Teixeira Gaspar
* António José Morim Brandão
* Cármen Isabel Amador Francisco
* Élia Cristina Viegas Pedro
* Fernando Manuel Antunes Marques da Silva
* Filipe Ferreira Cardoso Leitão
* jcm
* jcr-rep-entidade
* João Paulo de Melo Esteves Pereira
* Maria Leonor da Conceição Fresco Franco
* Maria Matos de Almeida Talhada Correia
* Nuno Filipe Casas Novas
* Regina Neves Lopes
* Ricardo Almeida
* Rui Araújo Entidade
* Rui Araújo Simples
* Sandra Cristina Patrício da Silva
* Sónia Isabel Ferreira Gonçalves Negrão
* Sónia Patrícia Pinheiro Reis

Nível 2 : 12%
* Aluno de DAW2020
* Carlos Matoso
* David Ferreira
* DAW2020-teste
* Joana Braga
* octavio
* PRI2020-teste

Nível 3 : 2%
* Ricardo Canela

Nível 3.5 : 2%
* Formação DGLAB

Nível 4 : 14%
* Cátia João Matias Trindade
* Cátia Trindade
* clara cristina rainho viegas
* Filipa Carvalho
* Madalena Ribeiro
* Maria Celeste Pereira
* Maria José Maciel Chaves
* Zélia Gomes

Nível 5 : 3%
* Pedro Penteado
* Silvestre Lacerda

Nível 6 : 3%
* Madalena Ribeiro
* Maria Rita Gago

Nível 7 : 26%
* Alexadre Teixeira
* Alexandra Lourenço
* Alexandre Teixeira
* Carlos Barbosa
* CLAV-migrator
* Design-DGLAB
* Duarte Freitas
* Frederico Pinto
* João Pimentel
* José Carlos Leite Ramalho
* José Carlos Martins
* Manuel Monteiro
* Miguel Ferreira
* Paulo Lima
* Rui Araújo

Pressione Enter!
>>  
\end{verbatim}


\subsection{Utilizadores agrupados por entidade, ordenados por nome e entidade.}

\begin{verbatim}
>> 4

*** Utilizadores agrupados por entidade ***

ent_A3ES:
* Alexandra Testes
* Alexandre Teixeira
* Aluno de DAW2020
* Carlos Barbosa
* CLAV-migrator
* DAW2020-teste
* Duarte Freitas
* jcr-rep-entidade
* João Pimentel
* José Carlos Leite Ramalho
* José Carlos Martins
* Manuel Monteiro
* octavio
* PRI2020-teste

ent_AAN:
* jcm
* Rui Araújo Entidade

ent_ACSS:
* Frederico Pinto

ent_APA:
* João Paulo de Melo Esteves Pereira

ent_BdP:
* Ricardo Canela

ent_CCDR:
* Ana Lúcia Cabrita Guerreiro
* Élia Cristina Viegas Pedro
* Maria Leonor da Conceição Fresco Franco
* Nuno Filipe Casas Novas

ent_CMABF:
* Sónia Isabel Ferreira Gonçalves Negrão

ent_CMSNS:
* Cármen Isabel Amador Francisco
* Sandra Cristina Patrício da Silva

ent_CMSPS:
* Filipe Ferreira Cardoso Leitão

ent_DGEG:
* Ricardo Almeida

ent_DGLAB:
* Alexandra Lourenço
* Cátia João Matias Trindade
* Cátia Trindade
* clara cristina rainho viegas
* Design-DGLAB
* Filipa Carvalho
* Formação DGLAB
* José Carlos Leite Ramalho
* Madalena Ribeiro
* Madalena Ribeiro
* Maria Celeste Pereira
* Maria José Maciel Chaves
* Maria Rita Gago
* Pedro Penteado
* Silvestre Lacerda
* Zélia Gomes

ent_FLUL:
* Alda do Carmo Namora Soares de Andrade

ent_ICNF:
* Maria Matos de Almeida Talhada Correia
* Sónia Patrícia Pinheiro Reis

ent_IEFP:
* Carlos Matoso
* David Ferreira
* Joana Braga

ent_II:
* Rui Araújo

ent_KEEP:
* Alexadre Teixeira
* Miguel Ferreira
* Paulo Lima

ent_LNEC:
* Rui Araújo Simples

ent_MdP:
* António José Morim Brandão

ent_SGMF:
* Ana Maria Teixeira Gaspar
* Regina Neves Lopes

ent_STI:
* Fernando Manuel Antunes Marques da Silva

ent_UTAD:
* Alexandra Maria Alves Coutinho Rodrigues


Pressione Enter!
>> 
\end{verbatim}



\subsection{Mostrar alguns indicadores.}

\begin{verbatim}
>> 5

*** Indicadores ***

Número de Utilizadores: 58

Número de Entidades: 21

Distribuição de utilizadores por entidade:
* ent_A3ES : 14
* ent_AAN : 2
* ent_ACSS : 1
* ent_APA : 1
* ent_BdP : 1
* ent_CCDR : 4
* ent_CMABF : 1
* ent_CMSNS : 2
* ent_CMSPS : 1
* ent_DGEG : 1
* ent_DGLAB : 16
* ent_FLUL : 1
* ent_ICNF : 2
* ent_IEFP : 3
* ent_II : 1
* ent_KEEP : 3
* ent_LNEC : 1
* ent_MdP : 1
* ent_SGMF : 2
* ent_STI : 1
* ent_UTAD : 1

Distribuição de utilizadores por nível:
* Nível 1 : 23
* Nível 2 : 7
* Nível 3 : 1
* Nível 3.5 : 1
* Nível 4 : 8
* Nível 5 : 2
* Nível 6 : 2
* Nível 7 : 16

Pressione Enter!
>> 
\end{verbatim}



\subsection{Imprimir num ficheiro Json os 20 primeiros registos.}

\begin{verbatim}
>> 6   

Digite nome do ficheiro de output!
>> json.txt

Ficheiro "json.txt" gerado com sucesso!

Pressione Enter!
>> 
\end{verbatim}

\textbf{Conteúdo do ficheiro json.txt}

\begin{verbatim}
{
  "registos":[
    {
       "utilizador":"Élia Cristina Viegas Pedro",
       "email":"epedro@ccdr-alg.pt",
       "entidade":"ent_CCDR",
       "nível de acesso":"1",
       "número de chamadas ao backend":"0"
    },
    {
       "utilizador":"Formação DGLAB",
       "email":"lurdes.almeida@dglab.gov.pt",
       "entidade":"ent_DGLAB",
       "nível de acesso":"3.5",
       "número de chamadas ao backend":"0"
    },
    {
       "utilizador":"Nuno Filipe Casas Novas",
       "email":"nuno.novas@ccdr-lvt.pt",
       "entidade":"ent_CCDR",
       "nível de acesso":"1",
       "número de chamadas ao backend":"0"
    },
    {
       "utilizador":"Sónia Patrícia Pinheiro Reis",
       "email":"sonia.reis@icnf.pt",
       "entidade":"ent_ICNF",
       "nível de acesso":"1",
       "número de chamadas ao backend":"0"
    },
    {
       "utilizador":"Sónia Isabel Ferreira Gonçalves Negrão",
       "email":"sonia.negrao@cm-albufeira.pt",
       "entidade":"ent_CMABF",
       "nível de acesso":"1",
       "número de chamadas ao backend":"0"
    },
    {
       "utilizador":"Filipe Ferreira Cardoso Leitão",
       "email":"arquivo@cm-spsul.pt",
       "entidade":"ent_CMSPS",
       "nível de acesso":"1",
       "número de chamadas ao backend":"0"
    },
    {
       "utilizador":"Ana Lúcia Cabrita Guerreiro",
       "email":"alucia@ccdr-alg.pt",
       "entidade":"ent_CCDR",
       "nível de acesso":"1",
       "número de chamadas ao backend":"0"
    },
    {
       "utilizador":"Alda do Carmo Namora Soares de Andrade",
       "email":"aandrade@letras.ulisboa.pt",
       "entidade":"ent_FLUL",
       "nível de acesso":"1",
       "número de chamadas ao backend":"0"
    },
    {
       "utilizador":"Ricardo Almeida",
       "email":"ricardo.almeida@dgeg.gov.pt",
       "entidade":"ent_DGEG",
       "nível de acesso":"1",
       "número de chamadas ao backend":"0"
    },
    {
       "utilizador":"Sandra Cristina Patrício da Silva",
       "email":"spatricio@mun-sines.pt",
       "entidade":"ent_CMSNS",
       "nível de acesso":"1",
       "número de chamadas ao backend":"0"
    },
    {
       "utilizador":"Cátia João Matias Trindade",
       "email":"catia.trindade@dglab.gov.pt",
       "entidade":"ent_DGLAB",
       "nível de acesso":"4",
       "número de chamadas ao backend":"0"
    },
    {
       "utilizador":"Ricardo Canela",
       "email":"tyty@tyty.pt",
       "entidade":"ent_BdP",
       "nível de acesso":"3",
       "número de chamadas ao backend":"0"
    },
    {
       "utilizador":"Cátia Trindade",
       "email":"matiasjcatia@gmail.com",
       "entidade":"ent_DGLAB",
       "nível de acesso":"4",
       "número de chamadas ao backend":"0"
    },
    {
       "utilizador":"Miguel Ferreira",
       "email":"mferreira@keep.pt",
       "entidade":"ent_KEEP",
       "nível de acesso":"7",
       "número de chamadas ao backend":"0"
    },
    {
       "utilizador":"Fernando Manuel Antunes Marques da Silva",
       "email":"fernando.marques.silva@marinha.pt",
       "entidade":"ent_STI",
       "nível de acesso":"1",
       "número de chamadas ao backend":"0"
    },
    {
       "utilizador":"Ana Maria Teixeira Gaspar",
       "email":"ana.gaspar@sgmf.gov.pt",
       "entidade":"ent_SGMF",
       "nível de acesso":"1",
       "número de chamadas ao backend":"0"
    },
    {
       "utilizador":"Maria Matos de Almeida Talhada Correia",
       "email":"MariaMatos.Correia@icnf.pt",
       "entidade":"ent_ICNF",
       "nível de acesso":"1",
       "número de chamadas ao backend":"0"
    },
    {
       "utilizador":"Cármen Isabel Amador Francisco",
       "email":"carmem@mun-sines.pt",
       "entidade":"ent_CMSNS",
       "nível de acesso":"1",
       "número de chamadas ao backend":"0"
    },
    {
       "utilizador":"Maria Leonor da Conceição Fresco Franco",
       "email":"leonor.mina@ccdr-lvt.pt",
       "entidade":"ent_CCDR",
       "nível de acesso":"1",
       "número de chamadas ao backend":"0"
    },
    {
       "utilizador":"Maria Rita Gago",
       "email":"m-rita.gago@dglab.gov.pt",
       "entidade":"ent_DGLAB",
       "nível de acesso":"6",
       "número de chamadas ao backend":"3"
    }
  ]
}

\end{verbatim}


\section{Ficheiro: 'forum-online.txt'}
\subsection{Conteúdo do ficheiro.}

\begin{verbatim}
registado777, registado777@gmail.com, ent_C, 2, 1
Dolores , mae_coragem@yahoo.com, ent_V, 2, 7
Prof. Marcelo, marselfie@hotmail.com, ent_P, 1, 2
CR7, siiiiiim@hotmail.com, ent_D, 7, 7
Eng. F. Santos, mete_mais_um_trinco@protonmail.com , ent_D, 3, 10
Xico Esperto, o.xico.esperto@protonmail.com, ent_C, 4 ,  7
Dr. Salgado, ddt@protonmail.com, ent_D, 2, 30
Sr. Doutor, sedutor@gmail.com, ent_D, 5, 1
Passos Dias Aguiar Mota, pdamota@gmail.com, ent_C, 8, 2
JJ, peaners@hotmail.com, ent_D, 3, 88
Zé Povinho, zezinho@gmail.com, ent_C, 2, 0
Socrates , animal_feroz@yahoo.pt, ent_P, 3, 2
Inacio, naosejasinacio@hotmail.com, ent_C, 4, 2
Geremias, geremeias666@yahoo.pt, ent_C, 4, 3
ativista, a.t.i.v.i.s.t.a@hotmail.com, ent_C, 2, 3
Baltasar Sete-Sóis, b72@hotmail.com, ent_V, 6, 1
geringonça, passarola@outlook.pt, ent_P, 4, 9
abelha maia, maia_a_abelha@yahoo.com, ent_V, 3, 4
Blimunda Sete-Luas, b_7_l@gmail.com, ent_V, 8, 1
mister_vitinho, i_speak_with-the-true@gmail.com, ent_D, 4, 8
o_faneca , faneca_xpto@yahoo.com, ent_C, 2, 7
Peppa , peppa_pig@gmail.com, ent_V, 5, 1 
\end{verbatim}


\subsection{Execução e menu.}
\begin{scriptsize}
\begin{verbatim}
>> python3 main.py forum-online.txt

*** Selecione Opção ***

1. Listagem com o nome e a entidade do utilizador, ordenada alfabeticamente por nome.
2. Lista ordenada alfabeticamente das entidades referenciadas, indicando, para cada uma, quantos utilizadores estão registados.
3. Distribuição de utilizadores por níveis de acesso.
4. Utilizadores, agrupados por entidade, ordenada primeiro pela entidade e dentro desta pelo nome.
5. Mostrar alguns indicadores.
6. Imprimir os 20 primeiros registos num novo ficheiro de output mas em formato jason.
0. Sair.
>> 
\end{verbatim}
\end{scriptsize}

\subsection{Nome dos utilizadores ordenados e respetiva entidade.}

\begin{verbatim}
>> 1

*** Utilizador : Entidade(s) ***

abelha maia : ent_V
ativista : ent_C
Baltasar Sete-Sóis : ent_V
Blimunda Sete-Luas : ent_V
CR7 : ent_D
Dolores : ent_V
Dr. Salgado : ent_D
Eng. F. Santos : ent_D
Geremias : ent_C
geringonça : ent_P
Inacio : ent_C
JJ : ent_D
mister_vitinho : ent_D
o_faneca : ent_C
Passos Dias Aguiar Mota : ent_C
Peppa : ent_V
Prof. Marcelo : ent_P
registado777 : ent_C
Socrates : ent_P
Sr. Doutor : ent_D
Xico Esperto : ent_C
Zé Povinho : ent_C

Pressione Enter!
>> 
\end{verbatim}


\subsection{Entidades ordenadas e número de utilizadores registados em cada uma.}

\begin{verbatim}
>> 2

*** Entidade : Nº Utilizadores ***

ent_C : 8
ent_D : 6
ent_P : 3
ent_V : 5

Pressione Enter!
>> 
\end{verbatim}



\subsection{Distribuiçao dos utilizadores por nível de acesso.}

\begin{verbatim}
>> 3

*** Nível de Acesso : Distribuição ***

Nível 1 : 5%
* Prof. Marcelo

Nível 2 : 27%
* ativista
* Dolores
* Dr. Salgado
* o_faneca
* registado777
* Zé Povinho

Nível 3 : 18%
* abelha maia
* Eng. F. Santos
* JJ
* Socrates

Nível 4 : 23%
* Geremias
* geringonça
* Inacio
* mister_vitinho
* Xico Esperto

Nível 5 : 9%
* Peppa
* Sr. Doutor

Nível 6 : 5%
* Baltasar Sete-Sóis

Nível 7 : 5%
* CR7

Nível 8 : 9%
* Blimunda Sete-Luas
* Passos Dias Aguiar Mota


Pressione Enter!
>> 
\end{verbatim}


\subsection{Utilizadores agrupados por entidade, ordenados por nome e entidade.}

\begin{verbatim}
>> 4

*** Utilizadores agrupados por entidade ***

ent_C:
* ativista
* Geremias
* Inacio
* o_faneca
* Passos Dias Aguiar Mota
* registado777
* Xico Esperto
* Zé Povinho

ent_D:
* CR7
* Dr. Salgado
* Eng. F. Santos
* JJ
* mister_vitinho
* Sr. Doutor

ent_P:
* geringonça
* Prof. Marcelo
* Socrates

ent_V:
* abelha maia
* Baltasar Sete-Sóis
* Blimunda Sete-Luas
* Dolores
* Peppa


Pressione Enter!
>> 
\end{verbatim}



\subsection{Mostrar alguns indicadores.}

\begin{verbatim}
>> 5

*** Indicadores ***

Número de Utilizadores: 22

Número de Entidades: 4

Distribuição de utilizadores por entidade:
* ent_C : 8
* ent_D : 6
* ent_P : 3
* ent_V : 5

Distribuição de utilizadores por nível:
* Nível 1 : 1
* Nível 2 : 6
* Nível 3 : 4
* Nível 4 : 5
* Nível 5 : 2
* Nível 6 : 1
* Nível 7 : 1
* Nível 8 : 2

Pressione Enter!
>> 
\end{verbatim}



\subsection{Imprimir num ficheiro Json os 20 primeiros registos.}

\begin{verbatim}
>> 6

Digite nome do ficheiro de output!
>> json.txt

Ficheiro "json.txt" gerado com sucesso!

Pressione Enter!
>> 
\end{verbatim}

\textbf{Conteúdo do ficheiro json.txt}

\begin{verbatim}
{
  "registos":[
    {
       "utilizador":"registado777",
       "email":"registado777@gmail.com",
       "entidade":"ent_C",
       "nível de acesso":"2",
       "número de chamadas ao backend":"1"
    },
    {
       "utilizador":"Dolores",
       "email":"mae_coragem@yahoo.com",
       "entidade":"ent_V",
       "nível de acesso":"2",
       "número de chamadas ao backend":"7"
    },
    {
       "utilizador":"Prof. Marcelo",
       "email":"marselfie@hotmail.com",
       "entidade":"ent_P",
       "nível de acesso":"1",
       "número de chamadas ao backend":"2"
    },
    {
       "utilizador":"CR7",
       "email":"siiiiiim@hotmail.com",
       "entidade":"ent_D",
       "nível de acesso":"7",
       "número de chamadas ao backend":"7"
    },
    {
       "utilizador":"Eng. F. Santos",
       "email":"mete_mais_um_trinco@protonmail.com",
       "entidade":"ent_D",
       "nível de acesso":"3",
       "número de chamadas ao backend":"10"
    },
    {
       "utilizador":"Xico Esperto",
       "email":"o.xico.esperto@protonmail.com",
       "entidade":"ent_C",
       "nível de acesso":"4",
       "número de chamadas ao backend":"7"
    },
    {
       "utilizador":"Dr. Salgado",
       "email":"ddt@protonmail.com",
       "entidade":"ent_D",
       "nível de acesso":"2",
       "número de chamadas ao backend":"30"
    },
    {
       "utilizador":"Sr. Doutor",
       "email":"sedutor@gmail.com",
       "entidade":"ent_D",
       "nível de acesso":"5",
       "número de chamadas ao backend":"1"
    },
    {
       "utilizador":"Passos Dias Aguiar Mota",
       "email":"pdamota@gmail.com",
       "entidade":"ent_C",
       "nível de acesso":"8",
       "número de chamadas ao backend":"2"
    },
    {
       "utilizador":"JJ",
       "email":"peaners@hotmail.com",
       "entidade":"ent_D",
       "nível de acesso":"3",
       "número de chamadas ao backend":"88"
    },
    {
       "utilizador":"Zé Povinho",
       "email":"zezinho@gmail.com",
       "entidade":"ent_C",
       "nível de acesso":"2",
       "número de chamadas ao backend":"0"
    },
    {
       "utilizador":"Socrates",
       "email":"animal_feroz@yahoo.pt",
       "entidade":"ent_P",
       "nível de acesso":"3",
       "número de chamadas ao backend":"2"
    },
    {
       "utilizador":"Inacio",
       "email":"naosejasinacio@hotmail.com",
       "entidade":"ent_C",
       "nível de acesso":"4",
       "número de chamadas ao backend":"2"
    },
    {
       "utilizador":"Geremias",
       "email":"geremeias666@yahoo.pt",
       "entidade":"ent_C",
       "nível de acesso":"4",
       "número de chamadas ao backend":"3"
    },
    {
       "utilizador":"ativista",
       "email":"a.t.i.v.i.s.t.a@hotmail.com",
       "entidade":"ent_C",
       "nível de acesso":"2",
       "número de chamadas ao backend":"3"
    },
    {
       "utilizador":"Baltasar Sete-Sóis",
       "email":"b72@hotmail.com",
       "entidade":"ent_V",
       "nível de acesso":"6",
       "número de chamadas ao backend":"1"
    },
    {
       "utilizador":"geringonça",
       "email":"passarola@outlook.pt",
       "entidade":"ent_P",
       "nível de acesso":"4",
       "número de chamadas ao backend":"9"
    },
    {
       "utilizador":"abelha maia",
       "email":"maia_a_abelha@yahoo.com",
       "entidade":"ent_V",
       "nível de acesso":"3",
       "número de chamadas ao backend":"4"
    },
    {
       "utilizador":"Blimunda Sete-Luas",
       "email":"b_7_l@gmail.com",
       "entidade":"ent_V",
       "nível de acesso":"8",
       "número de chamadas ao backend":"1"
    },
    {
       "utilizador":"mister_vitinho",
       "email":"i_speak_with-the-true@gmail.com",
       "entidade":"ent_D",
       "nível de acesso":"4",
       "número de chamadas ao backend":"8"
    }
  ]
}
\end{verbatim}

\section{Ficheiro: 'classroom.txt'}
\subsection{Conteúdo do ficheiro.}

\begin{verbatim}
Isaac Newton | a117293940@alunos.ufic.pt | ent_G12 | 1 | 0
Confúcio | a257394827@alunos.ufic.pt | ent_G09 | 1 | 0
Cristovão Colombo | a467392412@alunos.ufic.pt | ent_G12 | 1 | 0
J. R. R. Tolkien | a635485767@alunos.ufic.pt | ent_G05 | 1 | 2
Albert Einstein | a946596172@alunos.ufic.pt | ent_G14 | 1 | 1
Louis Pasteur | a327596378@alunos.ufic.pt | ent_G01 | 1 | 0
Galileu Galilei | a706203648@alunos.ufic.pt | ent_G18 | 1 | 0
Euclides de Alexandria | a954271239@alunos.ufic.pt | ent_G07 | 1 | 2
Charles Darwin | a227493822@alunos.ufic.pt | ent_G15 | 1 | 0
Nicolau Copérnico | a654642048@alunos.ufic.pt | ent_G02 | 1 | 0
Antoine Lavoisier | a804944252@alunos.ufic.pt | ent_G05 | 1 | 1
Michael Faraday | a375341811@alunos.ufic.pt | ent_G17 | 1 | 0
James Clerk Maxwell | a127394553@alunos.ufic.pt | ent_G08 | 1 | 1
Karl Marx | a677394914@alunos.ufic.pt | ent_G08 | 1 | 2
William Shakespeare | a9155647520@alunos.ufic.pt | ent_G02 | 1 | 0
Napoleão Bonaparte | a510264723@alunos.ufic.pt | ent_G19 | 2 | 1
Nikola Tesla | a477497937@alunos.ufic.pt | ent_G03 | 1 | 1
Platão de Atenas | a617293652@alunos.ufic.pt | ent_G04 | 1 | 1
Ludwig van Beethoven | a937597685@alunos.ufic.pt | ent_G16 | 1 | 1
Werner Heisenberg | a237691605@alunos.ufic.pt | ent_G00 | 1 | 0
Max Planck | a657792848@alunos.ufic.pt | ent_G04 | 1 | 0
Júlio César | a525495438@alunos.ufic.pt | ent_G13 | 1 | 1
Sigmund Freud | a922507686@alunos.ufic.pt | ent_G13 | 1 | 3
Voltaire  | a114503726@alunos.ufic.pt | ent_G19 | 2 | 1
Johannes Kepler | a529416472@alunos.ufic.pt | ent_G06 | 2 | 1
Nicolau Maquiavel | a375513115@alunos.ufic.pt | ent_G07 | 1 | 1
Vasco da Gama | a867395521@alunos.ufic.pt | ent_G14 | 1 | 2
Homero | a724466884@alunos.ufic.pt | ent_G17 | 1 | 1
Sidarta Gautama | a417093369@alunos.ufic.pt | ent_G20 | 1 | 0
Luís Vaz de Camões | a402207424@alunos.ufic.pt | ent_G20 | 1 | 0
Tony Hoare | a924364751@alunos.ufic.pt | ent_G21 | 1 | 1
Edsger Dijkstra | a821467613@alunos.ufic.pt | ent_G13 | 1 | 0
Marie Curie | a816385193@alunos.ufic.pt | ent_G10 | 1 | 3
Madre Teresa de Calcutá | a153257485@alunos.ufic.pt | ent_G00 | 1 | 1
Wolfgang Amadeus Mozart | a237292435@alunos.ufic.pt | ent_G03 | 1 | 1
René Descartes | a387294424@alunos.ufic.pt | ent_G03 | 1 | 0
Socrates | a705853823@alunos.ufic.pt | ent_G19 | 2 | 1
Arquimedes de Siracusa | a604152532@alunos.ufic.pt | ent_G11 | 1 | 0
Pierre de Fermat | a807293749@alunos.ufic.pt | ent_G01 | 1 | 2
Pitágoras de Samos  | a224612850@alunos.ufic.pt | ent_G10 | 1 | 1
Gottfried Wilhelm Leibniz | a887193477@alunos.ufic.pt | ent_G04 | 1 | 1
Carl Friedrich Gauss | a927394962@alunos.ufic.pt | ent_G11 | 1 | 0
Leonhard Euler | a721472365@alunos.ufic.pt | ent_G15 | 1 | 1
Joseph-Louis Lagrange | a657294866@alunos.ufic.pt | ent_G09 | 1 | 1
Augustin-Louis Cauchy | a390264536@alunos.ufic.pt | ent_G18 | 1 | 1
Daniel Bernoulli | a317395579@alunos.ufic.pt | ent_G16 | 1 | 1
Georg Cantor | a707293654@alunos.ufic.pt | ent_G16 | 1 | 1
Mahatma Gandhi | a826371829@alunos.ufic.pt | ent_G02 | 1 | 0
Nelson Mandela | a735598028@alunos.ufic.pt | ent_G21 | 1 | 3
Fyodor Dostoevsky | a127496527@alunos.ufic.pt | ent_G01 | 1 | 1
Liev Tolstói | a582250522@alunos.ufic.pt | ent_G06 | 2 | 3
Henry David Thoreau | a805315792@alunos.ufic.pt | ent_G00 | 1 | 1
\end{verbatim}


\subsection{Execução e menu.}
\begin{scriptsize}
\begin{verbatim}
>> python3 main.py classroom.txt

*** Selecione Opção ***

1. Listagem com o nome e a entidade do utilizador, ordenada alfabeticamente por nome.
2. Lista ordenada alfabeticamente das entidades referenciadas, indicando, para cada uma, quantos utilizadores estão registados.
3. Distribuição de utilizadores por níveis de acesso.
4. Utilizadores, agrupados por entidade, ordenada primeiro pela entidade e dentro desta pelo nome.
5. Mostrar alguns indicadores.
6. Imprimir os 20 primeiros registos num novo ficheiro de output mas em formato jason.
0. Sair.
>> 
\end{verbatim}
\end{scriptsize}

\subsection{Nome dos utilizadores ordenados e respetiva entidade.}

\begin{verbatim}
>> 1

*** Utilizador : Entidade(s) ***

Albert Einstein : ent_G14
Antoine Lavoisier : ent_G05
Arquimedes de Siracusa : ent_G11
Augustin-Louis Cauchy : ent_G18
Carl Friedrich Gauss : ent_G11
Charles Darwin : ent_G15
Confúcio : ent_G09
Cristovão Colombo : ent_G12
Daniel Bernoulli : ent_G16
Edsger Dijkstra : ent_G13
Euclides de Alexandria : ent_G07
Fyodor Dostoevsky : ent_G01
Galileu Galilei : ent_G18
Georg Cantor : ent_G16
Gottfried Wilhelm Leibniz : ent_G04
Henry David Thoreau : ent_G00
Homero : ent_G17
Isaac Newton : ent_G12
J. R. R. Tolkien : ent_G05
James Clerk Maxwell : ent_G08
Johannes Kepler : ent_G06
Joseph-Louis Lagrange : ent_G09
Júlio César : ent_G13
Karl Marx : ent_G08
Leonhard Euler : ent_G15
Liev Tolstói : ent_G06
Louis Pasteur : ent_G01
Ludwig van Beethoven : ent_G16
Luís Vaz de Camões : ent_G20
Madre Teresa de Calcutá : ent_G00
Mahatma Gandhi : ent_G02
Marie Curie : ent_G10
Max Planck : ent_G04
Michael Faraday : ent_G17
Napoleão Bonaparte : ent_G19
Nelson Mandela : ent_G21
Nicolau Copérnico : ent_G02
Nicolau Maquiavel : ent_G07
Nikola Tesla : ent_G03
Pierre de Fermat : ent_G01
Pitágoras de Samos : ent_G10
Platão de Atenas : ent_G04
René Descartes : ent_G03
Sidarta Gautama : ent_G20
Sigmund Freud : ent_G13
Socrates : ent_G19
Tony Hoare : ent_G21
Vasco da Gama : ent_G14
Voltaire : ent_G19
Werner Heisenberg : ent_G00
William Shakespeare : ent_G02
Wolfgang Amadeus Mozart : ent_G03

Pressione Enter!
>> 
\end{verbatim}


\subsection{Entidades ordenadas e número de utilizadores registados em cada uma.}

\begin{verbatim}
>> 2

*** Entidade : Nº Utilizadores ***

ent_G00 : 3
ent_G01 : 3
ent_G02 : 3
ent_G03 : 3
ent_G04 : 3
ent_G05 : 2
ent_G06 : 2
ent_G07 : 2
ent_G08 : 2
ent_G09 : 2
ent_G10 : 2
ent_G11 : 2
ent_G12 : 2
ent_G13 : 3
ent_G14 : 2
ent_G15 : 2
ent_G16 : 3
ent_G17 : 2
ent_G18 : 2
ent_G19 : 3
ent_G20 : 2
ent_G21 : 2

Pressione Enter!
>> 
\end{verbatim}



\subsection{Distribuiçao dos utilizadores por nível de acesso.}

\begin{verbatim}
>> 3

*** Nível de Acesso : Distribuição ***

Nível 1 : 90%
* Albert Einstein
* Antoine Lavoisier
* Arquimedes de Siracusa
* Augustin-Louis Cauchy
* Carl Friedrich Gauss
* Charles Darwin
* Confúcio
* Cristovão Colombo
* Daniel Bernoulli
* Edsger Dijkstra
* Euclides de Alexandria
* Fyodor Dostoevsky
* Galileu Galilei
* Georg Cantor
* Gottfried Wilhelm Leibniz
* Henry David Thoreau
* Homero
* Isaac Newton
* J. R. R. Tolkien
* James Clerk Maxwell
* Joseph-Louis Lagrange
* Júlio César
* Karl Marx
* Leonhard Euler
* Louis Pasteur
* Ludwig van Beethoven
* Luís Vaz de Camões
* Madre Teresa de Calcutá
* Mahatma Gandhi
* Marie Curie
* Max Planck
* Michael Faraday
* Nelson Mandela
* Nicolau Copérnico
* Nicolau Maquiavel
* Nikola Tesla
* Pierre de Fermat
* Pitágoras de Samos
* Platão de Atenas
* René Descartes
* Sidarta Gautama
* Sigmund Freud
* Tony Hoare
* Vasco da Gama
* Werner Heisenberg
* William Shakespeare
* Wolfgang Amadeus Mozart

Nível 2 : 10%
* Johannes Kepler
* Liev Tolstói
* Napoleão Bonaparte
* Socrates
* Voltaire

Pressione Enter!
>>  
\end{verbatim}


\subsection{Utilizadores agrupados por entidade, ordenados por nome e entidade.}

\begin{verbatim}
>> 4

*** Utilizadores agrupados por entidade ***

ent_G00:
* Henry David Thoreau
* Madre Teresa de Calcutá
* Werner Heisenberg

ent_G01:
* Fyodor Dostoevsky
* Louis Pasteur
* Pierre de Fermat

ent_G02:
* Mahatma Gandhi
* Nicolau Copérnico
* William Shakespeare

ent_G03:
* Nikola Tesla
* René Descartes
* Wolfgang Amadeus Mozart

ent_G04:
* Gottfried Wilhelm Leibniz
* Max Planck
* Platão de Atenas

ent_G05:
* Antoine Lavoisier
* J. R. R. Tolkien

ent_G06:
* Johannes Kepler
* Liev Tolstói

ent_G07:
* Euclides de Alexandria
* Nicolau Maquiavel

ent_G08:
* James Clerk Maxwell
* Karl Marx

ent_G09:
* Confúcio
* Joseph-Louis Lagrange

ent_G10:
* Marie Curie
* Pitágoras de Samos

ent_G11:
* Arquimedes de Siracusa
* Carl Friedrich Gauss

ent_G12:
* Cristovão Colombo
* Isaac Newton

ent_G13:
* Edsger Dijkstra
* Júlio César
* Sigmund Freud

ent_G14:
* Albert Einstein
* Vasco da Gama

ent_G15:
* Charles Darwin
* Leonhard Euler

ent_G16:
* Daniel Bernoulli
* Georg Cantor
* Ludwig van Beethoven

ent_G17:
* Homero
* Michael Faraday

ent_G18:
* Augustin-Louis Cauchy
* Galileu Galilei

ent_G19:
* Napoleão Bonaparte
* Socrates
* Voltaire

ent_G20:
* Luís Vaz de Camões
* Sidarta Gautama

ent_G21:
* Nelson Mandela
* Tony Hoare


Pressione Enter!
>> 
\end{verbatim}



\subsection{Mostrar alguns indicadores.}

\begin{verbatim}
>> 5

*** Indicadores ***

Número de Utilizadores: 52

Número de Entidades: 22

Distribuição de utilizadores por entidade:
* ent_G00 : 3
* ent_G01 : 3
* ent_G02 : 3
* ent_G03 : 3
* ent_G04 : 3
* ent_G05 : 2
* ent_G06 : 2
* ent_G07 : 2
* ent_G08 : 2
* ent_G09 : 2
* ent_G10 : 2
* ent_G11 : 2
* ent_G12 : 2
* ent_G13 : 3
* ent_G14 : 2
* ent_G15 : 2
* ent_G16 : 3
* ent_G17 : 2
* ent_G18 : 2
* ent_G19 : 3
* ent_G20 : 2
* ent_G21 : 2

Distribuição de utilizadores por nível:
* Nível 1 : 47
* Nível 2 : 5

Pressione Enter!
>>
\end{verbatim}



\subsection{Imprimir num ficheiro Json os 20 primeiros registos.}

\begin{verbatim}
>> 6   

Digite nome do ficheiro de output!
>> json.txt

Ficheiro "json.txt" gerado com sucesso!

Pressione Enter!
>> 
\end{verbatim}

\textbf{Conteúdo do ficheiro json.txt}

\begin{verbatim}
{
  "registos":[
    {
       "utilizador":"Isaac Newton",
       "email":"a117293940@alunos.ufic.pt",
       "entidade":"ent_G12",
       "nível de acesso":"1",
       "número de chamadas ao backend":"0"
    },
    {
       "utilizador":"Confúcio",
       "email":"a257394827@alunos.ufic.pt",
       "entidade":"ent_G09",
       "nível de acesso":"1",
       "número de chamadas ao backend":"0"
    },
    {
       "utilizador":"Cristovão Colombo",
       "email":"a467392412@alunos.ufic.pt",
       "entidade":"ent_G12",
       "nível de acesso":"1",
       "número de chamadas ao backend":"0"
    },
    {
       "utilizador":"J. R. R. Tolkien",
       "email":"a635485767@alunos.ufic.pt",
       "entidade":"ent_G05",
       "nível de acesso":"1",
       "número de chamadas ao backend":"2"
    },
    {
       "utilizador":"Albert Einstein",
       "email":"a946596172@alunos.ufic.pt",
       "entidade":"ent_G14",
       "nível de acesso":"1",
       "número de chamadas ao backend":"1"
    },
    {
       "utilizador":"Louis Pasteur",
       "email":"a327596378@alunos.ufic.pt",
       "entidade":"ent_G01",
       "nível de acesso":"1",
       "número de chamadas ao backend":"0"
    },
    {
       "utilizador":"Galileu Galilei",
       "email":"a706203648@alunos.ufic.pt",
       "entidade":"ent_G18",
       "nível de acesso":"1",
       "número de chamadas ao backend":"0"
    },
    {
       "utilizador":"Euclides de Alexandria",
       "email":"a954271239@alunos.ufic.pt",
       "entidade":"ent_G07",
       "nível de acesso":"1",
       "número de chamadas ao backend":"2"
    },
    {
       "utilizador":"Charles Darwin",
       "email":"a227493822@alunos.ufic.pt",
       "entidade":"ent_G15",
       "nível de acesso":"1",
       "número de chamadas ao backend":"0"
    },
    {
       "utilizador":"Nicolau Copérnico",
       "email":"a654642048@alunos.ufic.pt",
       "entidade":"ent_G02",
       "nível de acesso":"1",
       "número de chamadas ao backend":"0"
    },
    {
       "utilizador":"Antoine Lavoisier",
       "email":"a804944252@alunos.ufic.pt",
       "entidade":"ent_G05",
       "nível de acesso":"1",
       "número de chamadas ao backend":"1"
    },
    {
       "utilizador":"Michael Faraday",
       "email":"a375341811@alunos.ufic.pt",
       "entidade":"ent_G17",
       "nível de acesso":"1",
       "número de chamadas ao backend":"0"
    },
    {
       "utilizador":"James Clerk Maxwell",
       "email":"a127394553@alunos.ufic.pt",
       "entidade":"ent_G08",
       "nível de acesso":"1",
       "número de chamadas ao backend":"1"
    },
    {
       "utilizador":"Karl Marx",
       "email":"a677394914@alunos.ufic.pt",
       "entidade":"ent_G08",
       "nível de acesso":"1",
       "número de chamadas ao backend":"2"
    },
    {
       "utilizador":"William Shakespeare",
       "email":"a9155647520@alunos.ufic.pt",
       "entidade":"ent_G02",
       "nível de acesso":"1",
       "número de chamadas ao backend":"0"
    },
    {
       "utilizador":"Napoleão Bonaparte",
       "email":"a510264723@alunos.ufic.pt",
       "entidade":"ent_G19",
       "nível de acesso":"2",
       "número de chamadas ao backend":"1"
    },
    {
       "utilizador":"Nikola Tesla",
       "email":"a477497937@alunos.ufic.pt",
       "entidade":"ent_G03",
       "nível de acesso":"1",
       "número de chamadas ao backend":"1"
    },
    {
       "utilizador":"Platão de Atenas",
       "email":"a617293652@alunos.ufic.pt",
       "entidade":"ent_G04",
       "nível de acesso":"1",
       "número de chamadas ao backend":"1"
    },
    {
       "utilizador":"Ludwig van Beethoven",
       "email":"a937597685@alunos.ufic.pt",
       "entidade":"ent_G16",
       "nível de acesso":"1",
       "número de chamadas ao backend":"1"
    },
    {
       "utilizador":"Werner Heisenberg",
       "email":"a237691605@alunos.ufic.pt",
       "entidade":"ent_G00",
       "nível de acesso":"1",
       "número de chamadas ao backend":"0"
    }
  ]
}
\end{verbatim}




%%%%%%%%%%%%%%%%%%%%%%
%%%%%Capitulo5%%%%%%%%
%%%%%%%%%%%%%%%%%%%%%%

\chapter{Conclusão} \label{concl}
Com o projeto concluído esperamos ter cumprido todos os requisitos que nos foram propostos e que o nosso programa respeite os princípios ensinados na disciplina.
Numa primeira abordagem tentamos fazer a captura dos nomes de utilizador com a função \verb findall , no entanto não estava a ser muito fácil obter resultados na captura dos nomes de utilizador,por causa do inicio de linha, pelo que optamos por em todos os exercios iterar linha a linha. \\ \\
No problema que nos foi proposto, em cada uma das alineas é nos pedido que façamos um programa, no entanto achamos que a utilização do menu, na qual podemos escolher qual a funcionaliadade, torna a experiencia do utilizador mais simples e agrádavel. \\ \\
Todos concordamos que o facto de o projecto ter sido desenvolvido na linguagem 'Python' e com recurso à biblioteca 're', facilitou bastante a sua implementação.

%%%%%%%%%%%%%%%%%%%%%%
%%%%%Capitulo6%%%%%%%%
%%%%%%%%%%%%%%%%%%%%%%

\appendix % apendice
\chapter{Código do Programa}

\textbf{Ficheiro main.py}
\begin{scriptsize}
\begin{verbatim}
import re
import unidecode
import sys
import os.path


def name_entity_list():
    result = {}
    for line in text:
        user =  re.match(user_er,line).group()
        entity = re.search(entity_er,line).group()
        if user not in result:
            result[user] = [entity]
        else:
            result[user].append(', ')
            result[user].append(entity)
    result = list(result.items())
    result.sort(key=lambda x: unidecode.unidecode(x[0].casefold()))
    print("\n*** Utilizador : Entidade(s) ***\n")
    for r in result:
        print(r[0], ":", "".join(r[1]))


def entity_num_elements_list():
    entities = {}
    for line in text:
        entity = re.search(entity_er, line).group()
        if entity not in entities:
            entities[entity] = 1
        else:
            entities[entity] +=1

    result = list(entities.items())
    result.sort()
    print("\n*** Entidade : Nº Utilizadores ***\n")
    for r in result:
        print(r[0], ":", r[1])


def dist_users_level():
    levels = {}
    users = set()
    for line in text:
        broken_line = re.split(separator_er, line)
        user = re.match(user_er, broken_line[0]).group()
        level = re.search(level_er,broken_line[3]).group()
        if level not in levels:
            levels[level] = set()
            levels[level].add(user)
        else:
            levels[level].add(user)
        users.add(user)
    result = list(levels.items())
    result.sort()
    total_users = len(users)
    print("\n*** Nível de Acesso : Distribuição ***\n")
    for r in result:
        print(f"Nível {r[0]} : {round((len(r[1])/total_users)*100)}%" )
        for user in sorted(r[1], key=lambda x: unidecode.unidecode(x.casefold())):
            print("*",user)
        if(result.index(r) != len(result)-1):
            print("")


def name_entity_group():
    result = {}
    for line in text:
        user = re.match(user_er, line).group()
        entity = re.search(entity_er, line).group()
        if entity not in result:
            result[entity] = [user]
        else:
            result[entity].append(user)

    entities = list(result.keys())
    entities.sort()
    print("\n*** Utilizadores agrupados por entidade ***\n")
    for entity in entities:
        print(f"{entity}:")
        result[entity].sort(key=lambda x: unidecode.unidecode(x.casefold()))
        for user in result[entity]:
            print("*",user)
        print("")


def indicators():
    users = set()
    entities = {}
    levels = {}
    for line in text:
        broken_line = re.split(r'[:,|]+', line)
        users.add(re.match(user_er,broken_line[0]).group())
        entity = re.search(entity_er, broken_line[2]).group()
        level = re.search(level_er, broken_line[3]).group()

        if entity not in entities:
            entities[entity] = 1
        else:
            entities[entity] += 1
        if level not in levels:
            levels[level] = 1
        else:
            levels[level] += 1

    print("\n*** Indicadores ***\n")
    print("Número de Utilizadores:",len(users))
    print("")
    print("Número de Entidades:",len(entities.keys()))
    print("")
    print("Distribuição de utilizadores por entidade:")
    for entity in sorted(entities.keys()):
        print(f"* {entity} : {entities[entity]}")
    print("")
    print("Distribuição de utilizadores por nível:")
    for level in sorted(levels.keys()):
        print(f"* Nível {level} : {levels[level]}")


def json_20():
    list = []
    N = len(text)
    if N >= 20:
        N = 20

    for i in range(N):
        broken_line = re.split(separator_er, text[i])
        user = re.match(user_er, broken_line[0]).group()
        email = re.search(email_er, broken_line[1]).group()
        entity = re.search(entity_er, broken_line[2]).group()
        level = re.search(level_er, broken_line[3]).group()
        calls = re.search(calls_er, broken_line[4]).group()
        list.append((user,email,entity,level,calls))

    file_name = input("\nDigite nome do ficheiro de output!\n>> ")
    fp = open(file_name, 'w')

    fp.write("{\n\t\"registos\":[\n")
    for i in range(len(list)):
        l = list[i]
        fp.write("\t\t{\n")
        fp.write(f"\t\t\t \"utilizador\":\"{l[0]}\",\n")
        fp.write(f"\t\t\t \"email\":\"{l[1]}\",\n")
        fp.write(f"\t\t\t \"entidade\":\"{l[2]}\",\n")
        fp.write(f"\t\t\t \"nível de acesso\":\"{l[3]}\",\n")
        fp.write(f"\t\t\t \"número de chamadas ao backend\":\"{l[4]}\"\n")
        if i != 19:
            fp.write("\t\t},\n")
        else:
            fp.write("\t\t}\n")
    fp.write("\t]\n}\n")
    fp.close()
    print(f"\nFicheiro \"{file_name}\" gerado com sucesso!")


def menu():
    cls = lambda: print('\n' * 50)
    inputfromuser = ""
    options = []
    options.append("Listagem com o nome e a entidade do utilizador, ordenada alfabeticamente por nome.")
    options.append("Lista ordenada alfabeticamente das entidades referenciadas, indicando, para cada uma, quantos utilizadores estão registados.")
    options.append("Distribuição de utilizadores por níveis de acesso.")
    options.append("Utilizadores, agrupados por entidade, ordenada primeiro pela entidade e dentro desta pelo nome.")
    options.append("Mostrar alguns indicadores.")
    options.append("Imprimir os 20 primeiros registos num novo ficheiro de output mas em formato jason.")
    cls()
    while inputfromuser != '0':
        print("*** Selecione Opção ***\n")
        for i in range(len(options)):
            print(f"{i+1}. {options[i]}")
        print("0. Sair.")

        inputfromuser = input(">> ")
        while not inputfromuser.isdigit() or int(inputfromuser) > len(options) or int(inputfromuser) < 0:
            print("Opçao Invalida!")
            inputfromuser = input(">> ")

        if inputfromuser == '0':
            continue
        elif inputfromuser == '1':
            name_entity_list()
        elif inputfromuser == '2':
            entity_num_elements_list()
        elif inputfromuser == '3':
            dist_users_level()
        elif inputfromuser == '4':
            name_entity_group()
        elif inputfromuser == '5':
            indicators()
        else:
            json_20()

        input("\nPressione Enter!\n>> ")
        cls()


if len(sys.argv)>1:
    text_source = sys.argv[1]
else:
    text_source = 'clav-users.txt'

if not os.path.exists(text_source):
    print(f"Ficheiro \"{text_source}\" não encontrado!")
    sys.exit(0)

fp = open(text_source, 'r')
text = fp.readlines()
fp.close()

calls_er = r'\d+'
entity_er = r'ent_\w*'
email_er = r'(\w+|\.|@|_|-)+'
level_er = r'\d+\.?\d*'
separator_er = r'[:,|]+'
user_er = r'(\w+\.?\s*(-?\w+\.?\s*)*\b)'

menu()
\end{verbatim}
\end{scriptsize}




\lstinputlisting[caption={Exemplo de uma importação}, label={lstExe2}]{listagemImportadaLayout.l} %input de um ficheiro da listagem

%-- Fim do documento -- inserção das referencias bibliográficas

%\bibliographystyle{plain} % [1] Numérico pela ordem de citação ou ordem alfabetica
\bibliographystyle{alpha} % [Hen18] abreviação do apelido e data da publicação
%\bibliographystyle{apalike} % (Araujo, 2018) apelido e data da publicação
                             % --para usar este estilo descomente no inicio o comando \usepackage{apalike}

\bibliography{bibLayout} %input do ficheiro de referencias bibliograficas

\end{document} 